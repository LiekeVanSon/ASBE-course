\documentclass[11pt,a4paper]{report}

\usepackage{epsfig}
\usepackage{amsmath}
\usepackage{amssymb}
\usepackage{txfonts}

\textwidth 15.5cm
\textheight 23.0cm
\oddsidemargin 0cm
\evensidemargin 0cm
\topmargin 0cm

\pagestyle{empty}

\newcommand{\hints}[1]{\centerline{{\large\bf 
      Stellar Evolution -- Hints to exercises -- Chapter #1}} \bigskip
\renewcommand{\labelenumi}{#1.\arabic{enumi}}}

% for making counters of style 1.1, 1.2, etc
%\renewcommand{\labelenumi}{1.\arabic{enumi}}

% symbol definitions:
\newcommand{\Msun}{\ensuremath{{M}_\odot}}
\newcommand{\Lsun}{\ensuremath{{L}_\odot}}
\newcommand{\Rsun}{\ensuremath{{R}_\odot}}
\newcommand{\Mcore}{\ensuremath{M_\mathrm{c}}}
\newcommand{\MCh}{\ensuremath{M_\mathrm{Ch}}}
\newcommand{\Teff}{\ensuremath{T_\mathrm{eff}}}
\newcommand{\lum}{\ensuremath{l}}
\newcommand{\ledd}{\ensuremath{l_\mathrm{Edd}}}
\newcommand{\Ledd}{\ensuremath{L_\mathrm{Edd}}}
\newcommand{\Prad}{\ensuremath{P_\mathrm{rad}}}
\newcommand{\Pgas}{\ensuremath{P_\mathrm{gas}}}
\newcommand{\Pion}{\ensuremath{P_\mathrm{ion}}}
\newcommand{\Prot}{\ensuremath{P_\mathrm{rot}}}
\newcommand{\GammaC}{\ensuremath{{\Gamma_\mathrm{C}}}}
\newcommand{\gammad}{\ensuremath{{\gamma_\mathrm{ad}}}}
\newcommand{\gradad}{\ensuremath{{\nabla_\mathrm{ad}}}}
\newcommand{\gradrad}{\ensuremath{{\nabla_\mathrm{rad}}}}
\newcommand{\tdyn}{\ensuremath{{\tau_\mathrm{dyn}}}}
\newcommand{\tKH}{\ensuremath{{\tau_\mathrm{KH}}}}
\newcommand{\tnuc}{\ensuremath{{\tau_\mathrm{nuc}}}}
\newcommand{\tms}{\ensuremath{{\tau_\mathrm{MS}}}}
\newcommand{\enuc}{\ensuremath{{\epsilon_\mathrm{nuc}}}}
\newcommand{\eneu}{\ensuremath{{\epsilon_\nu}}}
\newcommand{\egrav}{\ensuremath{{\epsilon_\mathrm{gr}}}}
\newcommand{\cP}{\ensuremath{{c_{\scriptstyle P}}}}
\newcommand{\cV}{\ensuremath{{c_{\scriptstyle V}}}}
\newcommand{\lmix}{\ensuremath{{\ell_\mathrm{m}}}}
\newcommand{\HP}{\ensuremath{{H_{\scriptstyle P}}}}
\newcommand{\vel}{\ensuremath{{\upsilon}}}
\newcommand{\vconv}{\ensuremath{{\vel_\mathrm{c}}}}
\newcommand{\vsound}{\ensuremath{{\vel_\mathrm{s}}}}
\newcommand{\veq}{\ensuremath{{\vel_\mathrm{eq}}}}
\newcommand{\vect}[1]{\mbox{\boldmath $#1$}}
\newcommand{\gradient}{\mbox{\boldmath $\nabla$}}

\newcommand{\Chi}{\ensuremath{\raisebox{0.4ex}{$\chi$}}}

\newcommand{\kmsMpc}{\ensuremath{%
  \mathrm{km}\,\mathrm{s}^{-1}\,\mathrm{Mpc}^{-1}}}

\newcommand{\arrow}{\ensuremath{\Rightarrow}}
\newcommand{\lsim}{\mathrel{\hbox{\rlap{\lower.55ex \hbox {$\sim$}}
 \kern-.3em \raise.4ex \hbox{$<$}}}}
\newcommand{\gsim}{\mathrel{\hbox{\rlap{\lower.55ex \hbox {$\sim$}}
 \kern-.3em \raise.4ex \hbox{$>$}}}}

\newcommand{\beq}{\begin{equation}}
\newcommand{\eeq}{\end{equation}}
\newcommand{\beqar}{\begin{eqnarray}}
\newcommand{\eeqar}{\end{eqnarray}}
\newcommand{\der}{\ensuremath{\mathrm{d}}}
\newcommand{\dd}{\partial}

\newcommand{\half}{\ensuremath{{\textstyle\frac{1}{2}}}}
\newcommand{\third}{\ensuremath{{\textstyle\frac{1}{3}}}}
\newcommand{\onesixth}{\ensuremath{{\textstyle\frac{1}{6}}}}

\newcommand{\amu}{\ensuremath{m_\mathrm{u}}}
\newcommand{\e}{\ensuremath{\mathrm{e}}}
\newcommand{\n}{\ensuremath{\mathrm{n}}}
\newcommand{\p}{\ensuremath{\mathrm{p}}}
\renewcommand{\H}[1]{\ensuremath{{^{#1}\mathrm{H}}}}
\newcommand{\D}{\ensuremath{\mathrm{D}}}
\newcommand{\He}[1]{\ensuremath{{^{#1}\mathrm{He}}}}
\newcommand{\Li}[1]{\ensuremath{{^{#1}\mathrm{Li}}}}
\newcommand{\Be}[1]{\ensuremath{{^{#1}\mathrm{Be}}}}
\newcommand{\B}[1]{\ensuremath{{^{#1}\mathrm{B}}}}
\newcommand{\C}[1]{\ensuremath{{^{#1}\mathrm{C}}}}
\newcommand{\N}[1]{\ensuremath{{^{#1}\mathrm{N}}}}
\renewcommand{\O}[1]{\ensuremath{{^{#1}\mathrm{O}}}}
\newcommand{\F}[1]{\ensuremath{{^{#1}\mathrm{F}}}}
\newcommand{\Na}[1]{\ensuremath{{^{#1}\mathrm{Na}}}}
\newcommand{\Ne}[1]{\ensuremath{{^{#1}\mathrm{Ne}}}}
\newcommand{\Mg}[1]{\ensuremath{{^{#1}\mathrm{Mg}}}}
\newcommand{\Al}[1]{\ensuremath{{^{#1}\mathrm{\!Al}}}}
\newcommand{\Si}[1]{\ensuremath{{^{#1}\mathrm{Si}}}}
\newcommand{\Fe}[1]{\ensuremath{{^{#1}\mathrm{Fe}}}}
\newcommand{\el}[2]{\ensuremath{{^{#2}\mathrm{#1}}}}
\newcommand{\avg}[1]{\ensuremath{\langle#1\rangle}}
\newcommand{\rate}[1]{\ensuremath{r_\mathrm{#1}}}
\newcommand{\sv}[1]{\ensuremath{\langle\sigma v\rangle_{#1}}}
\newcommand{\svpp}{\ensuremath{\langle\sigma v\rangle_\mathrm{pp}}}
\newcommand{\svpD}{\ensuremath{\langle\sigma v\rangle_\mathrm{pD}}}
\newcommand{\life}[2]{\ensuremath{\tau_\mathrm{#1}(\mathrm{#2})}}
\newcommand{\fpp}[1]{\ensuremath{f_\mathrm{pp#1}}}


\begin{document}

\hints{3}

\begin{enumerate}

\item {\bf Conceptual questions}

  \begin{enumerate}

  \item

    Read parahtaph 3.1 of the lecture notes.
    
  \item 

    Degenareracy is very important in neutron stars, white dwarfs and
    in the cores of red giants. It also plays a role in low-mass main
    sequence stars.

  \item

    See the left panel of Figure~3.2 of the lecture notes. When the
    density increases the electrons are `pushed' to higher $p$-states,
    because of their higher momenta they exert more pressure.

    
  \item

    $p_{\rm F}$ increases when the density increases.  As $\vel_{\max}
    = p_{\rm F}/m_{\rm e}$ approaches $c$, the particles become
    relativistic. Also read paragraph 3.3.5 of the lecture notes.

  \item

    Read the final section of paragraph 3.3.5: importance of electron
    degeneracy in stars.
    
  \end{enumerate}

\item {\bf Mean molecular weight} 

  See Section~3.3.3.  The mean molecular weight $\mu$ is the average
  mass (in \amu) per particle. Assume $m_e \approx 0$, and assume full
  ionization. The total number of particles contributed by an atom of
  species $i$ is $N_i = Z_i + 1$. Let $n_i$ be the number/cm$^3$ of
  nuclei of species $i$. Then

  \[
  \mu = \frac{\text{total mass/\amu}}{\text{number of particles}} =
  \frac{\sum_i A_i n_i}{\sum N_i n_i} = \frac{\sum_i X_i}{\sum (1+Z_i)
    X_i/A_i}
  \]

  which gives eq.~(3.23).

\item  {\bf The {\boldmath$\rho-T$} plane }

  Consider the equations of state for each of these regions: ideal gas
  $P_\mathrm{i}$ eq.~(3.21), radiation \Prad\ eq.~(3.44), NR degeneracy
  $P_\mathrm{NR}$ eq.~(3.35) and ER degeneracy $P_\mathrm{ER}$
  eq.~(3.37).

  For the transitions between the regions solve $P_\mathrm{i} =
  \Prad$, $P_\mathrm{i} = P_\mathrm{NR}$, $P_\mathrm{i} =
  P_\mathrm{ER}$ and $P_\mathrm{NR} = P_\mathrm{ER}$.  The solutions
  are given in Section~3.3.7.

\item {\bf The pressure of a gas of free particles}

  \begin{enumerate}

  \item See the start of Section~3.3.

  \item Follow the hints given in Section~3.3.2.

  \item See Section~3.3.1.

  \item See Section~3.3.4.

  \item See Section~3.3.5.

  \item See Section~3.3.5.

  \item See Section~3.3.5.

  \item Hint: use $\epsilon = pc$, $U = \int_0^\infty pc\, n(p)dp$ and
    $\displaystyle \int_0^\infty \frac{x^3}{\exp(x) - 1}\,dp =
    \frac{\pi^4}{15}$.

  \end{enumerate}

\newpage
\item {\bf Adiabatic derivatives}

  \begin{enumerate}

  \item For an ideal gas $U =\frac{3}{2}NkT$ and $P = NkT/V$.  Insert
    these in the first law of thermodynamics:

    \[
      dU = - PdV  \quad\Rightarrow\quad\
      d(\frac{3}{2}N k T) = -\frac{N k T}{V}dV  \quad\Rightarrow\quad\
      \frac{3}{2} \frac{dT}{T} = - \frac{dV}{V} \quad\Rightarrow\quad\
    \]
    \[
      d\ln T = -\frac{2}{3} d \ln V   \quad\Rightarrow\quad\
      T \propto V^{-\frac{2}{3}}   \quad\Rightarrow\quad\
      P V \propto V^{-\frac{2}{3}}  \quad\Rightarrow\quad\
      P \propto V^{-\frac{5}{3}}  \quad\Rightarrow\quad\
    \]
    \begin{equation}
      P \propto \rho^{\frac{5}{3}}
    \end{equation}
   
  \item 
    \begin{equation}
      P  \propto \rho^{\frac{5}{3}} \Rightarrow
      P  \propto \frac{P}{T}^{\frac{5}{3}} \Rightarrow
      T  \propto P^{\frac{2}{5}} \Rightarrow
      d \ln T = \frac{2}{5} d \ln P \Rightarrow
      \frac{d \ln T}{d \ln P} = \frac{2}{5}
    \end{equation}

  \item 

    $\displaystyle \nabla_{\rm ad} = \left( \frac{d \log T}{d \log
        P}\right)_{\rm ad}$ can be interpreted as a gradient because
    $\log P$ increases monotonically inside a star in HE, so $P$ can
    be considered as a coordinate, like $r$ or $m$.

  \item Hints:
    \begin{itemize}
    \item Write down the equation of state for the mixture in
      differential form, eq.~(3.48), and express $\chi_T$ and
      $\chi_\rho$ in terms of $\beta$. This should give: ~ $d\log P =
      \beta\,d\log\rho + (4-3\beta)\,d\log T$.
    \item Write down the internal energy in differential form, and
      again express this in terms of $\beta$. This should give: ~
      $(\rho/P)\,du = (12 - \frac{21}{2}\beta)\,d\log T -
      3(1-\beta)\,d\log\rho$.
    \item Apply the first law of thermodynamics to derive the relation
      between $d\log T$ and $d\log\rho$ for an adiabatic process,
      again expressed in $\beta$.
    \item Finally eliminate $d\log T$ from the two relations.
    \end{itemize}
    
  \item $\beta = 0 \Rightarrow \gradad=\frac{4}{3}$ (radiation dominates) \\
    $\beta = 1 \Rightarrow \gradad=\frac{5}{3}$ (ideal gas pressure dominates) \\

  \end{enumerate}


\item {\bf Ionization effects }
  \begin{enumerate}
        
  \item The fraction $E_\mathrm{C}/kT$, eq.~(3.70) is $\ll 1$ for
    normal stars. In planets it is substantial and can even dominate
    (solidifaction). For a degenerate gas
    \[
    \frac{E_c}{p_{\rm F}^2/2m} \propto n^{-1/3}, 
    \]
    so the effect becomes less important for higher densities. Electrons
    in stars behave like an ideal gas, but the ions can form a
    crystallization in sufficiently cool WD.
    
    %The particles in an ionized gas are charged and therefore
    %    undergo electrostatic (Coulomb) interactions. Why can can we
    %    nevertheless make the ideal-gas assumption in most stars
    %    (i.e. that the internal energy of the gas is just the sum of the
    %    kinetic energies of the particles)? For which stars do Coulomb
    %    interactions have a significant effect?
    
  \item The Saha equation (which gives that the ions start to
    recombine at very high densities) is not valid for high densities.
    For very high densities the `potential wells' of the ions start to
    overlap, which leads to fewer bound states. Therefore the energy
    needed to kick an electron from a bound to a free state becomes
    lower and eventually zero.
    %Why does the gas in the interior of a star become
    %    pressure-ionized at high densities?
	
  \item See Section~3.5.1. By compressing an ideal gas adiabatically,
    the work done is all converted into heat. If the gas is partially
    ionized, a certain amount of the work done is used to ionize the
    gas further and only a part is left for heating the gas. In other
    words, a certain increase of pressure will lead to a larger
    temperature increase for an ideal gas than for an partially
    ionized gas.
    
    
    %Explain qualitatively why partial ionization leads to
    %    $\nabla_\mathrm{ad} < \nabla_\mathrm{ad,ideal} = 0.4$, in other
    %    words: why does adiabatic compression lead to to much smaller
    %    temperature increase when the gas is partly ionized, compared to a
    %    completely ionized (or unionized) gas?
    
\end{enumerate}
  


 \end{enumerate}





\end{document}




\item {\bf Summary of the Lecture}

 \begin{enumerate}

  \item - 
  \item You have to integrate an integral in the form $I_n(a) = \int
  x^n exp(-a x^2)$.  For the standard integral one can show that
  $I_0(a) = \sqrt{\pi a}$.  Show first that for $n \ge 1$
  
  \[
  I_n(a) = -\frac{dI_{n-1}}{da}.
  \]
  From this you will find that $I_1(a) = \frac{1}{2}\sqrt{pi}a^{-3/2}$
  and $I_2(a) = $....

  When your done you should find a nice simple familiar expression for
  the pressure.
  
  \item Use equation 3.5 $U = n <\epsilon_p>$ and 3.6 $P = \frac{1}{3}
  n <p v_p>$, with $\epsilon_p = \frac{p^2}{2m}$ and $v =
  \frac{p}{m}$.

  \item


  \item Imagine a volume $V$ which partioned in little cubes with size
  $\Delta x\Delta y\Delta z$.  There are $V/\Delta x\Delta y\Delta z$
  positions in this volume for an electron.  

  We can do a similar thing for the ``momentum space'' $V_p$, which we
  divide in cubes $\Delta p_x\Delta p_y\Delta p_z$.  We are interested
  in electrons with momentum between $p$ and $p + dp$, where $ p =
  \sqrt{p_x^2 +p_y^2+p_z^2 }$.  The volume $V_p$ is a spherical shell
  with radius $p$ and thickness $dp$.  Therefore $V_p = 4 \pi p^2
  dp$. The number of ``positions'' in the momentum space are
  $V_p/\Delta p_x\Delta p_y\Delta p_z $.

  The total number of postions is 
  \[
  \frac {V * V_p}{ \Delta x\Delta y\Delta z\Delta p_x\Delta p_y\Delta
  p_z } = \\ \frac {V * 4 \pi p^2 dp } {(\Delta x \Delta p_x)* (\Delta z
  \Delta p_z)*(\Delta z \Delta p_z) } = \\
  \frac {V * 4 \pi p^2 }{h^3}
  \]

  As electrons are fermions two of them can occupy the same space,
  therefore multiply by 2.

 \item -
 \item Read Equation 3.36 in the lecture notes.
 \item Read Equation 3.38 in the lecture notes.

 \item -
 \item -

\end{enumerate}


 \end{enumerate}





\newpage



{\bf IMPORTANT}: The solutions of the following problems are required
for understanding and carrying out Practicum 2 on Tuesday, the 20th of
February 2007.
 
\renewcommand{\labelenumi}{P.\arabic{enumi}}
\begin{enumerate}

\item {\bf Polytropes I} A number of interesting properties of
stars can be derived from simple stellar models in which the pressure
is a function of the density, the so-called polytropic models. For a
polytropic equation of state we can write
\[
P=K\rho^{\gamma}
\]
with $K$ and $\gamma$ constant. $\gamma$ is related to the so-called
polytropic index $n$, as $\gamma = 1+\frac{1}{n}$.

If the equation of state can be written in polytropic form, the mechanical
structure of a star is completely determined by the equations for mass
continuity ($dm/dr$) and hydrostatic equilibrium ($dP/dr$).\\
a) Show that the combination of these two differential equations
together with the polytropic equation of state gives the following
second-order differential equation for the density:
\[
\frac{1}{\rho r^2}\frac{d}{dr}\left(r^2\rho^{\gamma -
2}\frac{d\rho}{dr}\right)=-\frac{4\pi G}{K\gamma}
\]\\
(Hint: multiply the hydrostatic equation by $r^2/\rho$ and take the
derivative with respect to $r$). \\
In order to construct a stellar model, this equation has to be solved
together with two boundary conditions which are set at the center:
$\rho(0)=\rho_c \qquad$ and $\qquad
\left(d\rho/dr\right)_{r=0}=0$, where $\rho_c$ is a parameter
to be chosen.\\
b) What determines the second boundary condition, i.e., why does the
density gradient have to vanish at the center?\\
c) Derive $K$ and $\gamma$ for the equation of state of an ideal
gas at a fixed temperature $T$, of a non-relativistic degenerate gas and
of a relativistic degenerate gas.\\

\item {\bf Polytropes II} In order to simplify the equation of
derived in 3.3.a, we define two new dimensionless variables $w$
(related to the density) and $z$ (related to the radius) and write
\begin{eqnarray*}
& & \rho = \rho_c w^n,\\
& & r = \alpha z, \qquad \mathrm{with} \qquad
\alpha = \left(\frac{n+1}{4\pi G}K\rho_c^{1/n-1}\right)^{1/2}
\end{eqnarray*}
This choice of $\alpha$ ensures that the constants $K$ and $4\pi G$ are
eliminated after substituting $r$ and $\rho$ into the equation of problem
2.a. The resulting second-order differential equation is called the
Lane-Emden (L-E) equation:
\[
\frac{1}{z^2}\frac{d}{dz}\left(z^2\frac{dw}{dz}\right)+w^n=0.
\]
A stellar model can be constructed by integrating this equation outwards
from the center. The boundary conditions imply that in the center ($z=0$)
we have $w=1$ and $dw/dz=0$.  The integration is done until the surface is
reached, which is determined by the condition $\rho = 0$ at $r=R$, or in
terms of $w$ and $z$: $w(z_n) = 0$ at $z_n = R/\alpha$.\\
Hence a polytrope is uniquely determined by three parameters: $n$ (or
$\gamma$), $K$ and $\rho_c$.  These three parameters determine the radius
$R$ and mass $M$ of the polytropic star.\\
a) In general the L-E equation does not have an analytical solution,
but has to be solved numerically. Exceptions are $n=0$ and $N=1$. Solve
them analytically.\\
b) Using the L-E equation, show that the mass and the radius of a
polytropic star can be expressed as the following functions of $\rho_c$,
for given values of $n$ and $K$:
\begin{eqnarray*}
& & M=4\pi a^3 \rho_c\theta_n=4\pi \left[\frac{(n+1)K}{4\pi G}\right]^{3/2}
\rho_c^{\frac{3-n}{2n}}\theta_n\\
& & R=\alpha z_n=\left[\frac{(n+1)K}{4\pi G}\right]^{1/2}
\rho_c^{\frac{1-n}{2n}}z_n,
\end{eqnarray*}
where $\theta_n\equiv(-z^2dw/dz)_{z=z_n}$. The values of $\theta_n$ and
$z_n$ depend only on $n$ and are tabulated in Table 19.1 of K\&W.\\
c) Derive the central density $\rho_c$ and the central pressure $P_c$ as
function of its mass and radius, assuming that the solution of the
L-E equation is known (Table 19.1 of K\&W).\\

\item {\bf White dwarfs} Stars that are so compact and dense that
their interior pressure is dominated by degenerate electrons are known
observationally as \emph{white dwarfs} (WDs).They are the remnants of
stellar cores in which hydrogen has been completely converted into
helium and, in most cases, also helium has been fused into carbon and
oxygen.  To understand some of the properties of white dwarfs (WDs) we
start by considering the equation of state for a degenerate,
non-relativistic electron gas.\\ a) What is the value of $K$ for such
a star? Remember to consider an appropriate value of the mean
molecular weight per free electron $\mu_e$.\\ b) Derive how the
central density $\rho_c$ depends on the mass of a WD. Using this with
the result of problem 3.4.b, derive a radius-mass relation
$R=R(M)$. Interpret this physically.\\

The result above shows that more massive white dwarfs are more compact,
and therefore have a higher density. Above a certain density the electrons
become relativistic as they are pushed up to higher momenta by the Pauli
exclusion principle. The degree of relativity increases with density, and
therefore with the mass of the, until at a certain mass all the electrons
become extremely relativistic, i.e., their speed $v_e\rightarrow c$.\\
c) Use the result of exercise 3.5.b to estimate for which WD masses the
relativistic effects would become important.\\
d) Show that the derivation of a $R=R(M)$ relation for the extreme
relativistic case leads to a unique mass. This mass is the so-called
\emph{Chandrasekhar mass} ($M_{Ch}$). Calculate its value.
\end{enumerate}
