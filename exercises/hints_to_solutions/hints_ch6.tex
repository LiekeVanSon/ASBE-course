\documentclass[11pt,a4paper]{report}

\usepackage{epsfig}
\usepackage{amsmath}
\usepackage{amssymb}
\usepackage{txfonts}

\textwidth 15.5cm
\textheight 23.0cm
\oddsidemargin 0cm
\evensidemargin 0cm
\topmargin 0cm

\pagestyle{empty}

\newcommand{\hints}[1]{\centerline{{\large\bf 
      Stellar Evolution -- Hints to exercises -- Chapter #1}} \bigskip
\renewcommand{\labelenumi}{#1.\arabic{enumi}}}

% for making counters of style 1.1, 1.2, etc
%\renewcommand{\labelenumi}{1.\arabic{enumi}}

% symbol definitions:
\newcommand{\Msun}{\ensuremath{{M}_\odot}}
\newcommand{\Lsun}{\ensuremath{{L}_\odot}}
\newcommand{\Rsun}{\ensuremath{{R}_\odot}}
\newcommand{\Mcore}{\ensuremath{M_\mathrm{c}}}
\newcommand{\MCh}{\ensuremath{M_\mathrm{Ch}}}
\newcommand{\Teff}{\ensuremath{T_\mathrm{eff}}}
\newcommand{\lum}{\ensuremath{l}}
\newcommand{\ledd}{\ensuremath{l_\mathrm{Edd}}}
\newcommand{\Ledd}{\ensuremath{L_\mathrm{Edd}}}
\newcommand{\Prad}{\ensuremath{P_\mathrm{rad}}}
\newcommand{\Pgas}{\ensuremath{P_\mathrm{gas}}}
\newcommand{\Pion}{\ensuremath{P_\mathrm{ion}}}
\newcommand{\Prot}{\ensuremath{P_\mathrm{rot}}}
\newcommand{\GammaC}{\ensuremath{{\Gamma_\mathrm{C}}}}
\newcommand{\gammad}{\ensuremath{{\gamma_\mathrm{ad}}}}
\newcommand{\gradad}{\ensuremath{{\nabla_\mathrm{ad}}}}
\newcommand{\gradrad}{\ensuremath{{\nabla_\mathrm{rad}}}}
\newcommand{\tdyn}{\ensuremath{{\tau_\mathrm{dyn}}}}
\newcommand{\tKH}{\ensuremath{{\tau_\mathrm{KH}}}}
\newcommand{\tnuc}{\ensuremath{{\tau_\mathrm{nuc}}}}
\newcommand{\tms}{\ensuremath{{\tau_\mathrm{MS}}}}
\newcommand{\enuc}{\ensuremath{{\epsilon_\mathrm{nuc}}}}
\newcommand{\eneu}{\ensuremath{{\epsilon_\nu}}}
\newcommand{\egrav}{\ensuremath{{\epsilon_\mathrm{gr}}}}
\newcommand{\cP}{\ensuremath{{c_{\scriptstyle P}}}}
\newcommand{\cV}{\ensuremath{{c_{\scriptstyle V}}}}
\newcommand{\lmix}{\ensuremath{{\ell_\mathrm{m}}}}
\newcommand{\HP}{\ensuremath{{H_{\scriptstyle P}}}}
\newcommand{\vel}{\ensuremath{{\upsilon}}}
\newcommand{\vconv}{\ensuremath{{\vel_\mathrm{c}}}}
\newcommand{\vsound}{\ensuremath{{\vel_\mathrm{s}}}}
\newcommand{\veq}{\ensuremath{{\vel_\mathrm{eq}}}}
\newcommand{\vect}[1]{\mbox{\boldmath $#1$}}
\newcommand{\gradient}{\mbox{\boldmath $\nabla$}}

\newcommand{\Chi}{\ensuremath{\raisebox{0.4ex}{$\chi$}}}

\newcommand{\kmsMpc}{\ensuremath{%
  \mathrm{km}\,\mathrm{s}^{-1}\,\mathrm{Mpc}^{-1}}}

\newcommand{\arrow}{\ensuremath{\Rightarrow}}
\newcommand{\lsim}{\mathrel{\hbox{\rlap{\lower.55ex \hbox {$\sim$}}
 \kern-.3em \raise.4ex \hbox{$<$}}}}
\newcommand{\gsim}{\mathrel{\hbox{\rlap{\lower.55ex \hbox {$\sim$}}
 \kern-.3em \raise.4ex \hbox{$>$}}}}

\newcommand{\beq}{\begin{equation}}
\newcommand{\eeq}{\end{equation}}
\newcommand{\beqar}{\begin{eqnarray}}
\newcommand{\eeqar}{\end{eqnarray}}
\newcommand{\der}{\ensuremath{\mathrm{d}}}
\newcommand{\dd}{\partial}

\newcommand{\half}{\ensuremath{{\textstyle\frac{1}{2}}}}
\newcommand{\third}{\ensuremath{{\textstyle\frac{1}{3}}}}
\newcommand{\onesixth}{\ensuremath{{\textstyle\frac{1}{6}}}}

\newcommand{\amu}{\ensuremath{m_\mathrm{u}}}
\newcommand{\e}{\ensuremath{\mathrm{e}}}
\newcommand{\n}{\ensuremath{\mathrm{n}}}
\newcommand{\p}{\ensuremath{\mathrm{p}}}
\renewcommand{\H}[1]{\ensuremath{{^{#1}\mathrm{H}}}}
\newcommand{\D}{\ensuremath{\mathrm{D}}}
\newcommand{\He}[1]{\ensuremath{{^{#1}\mathrm{He}}}}
\newcommand{\Li}[1]{\ensuremath{{^{#1}\mathrm{Li}}}}
\newcommand{\Be}[1]{\ensuremath{{^{#1}\mathrm{Be}}}}
\newcommand{\B}[1]{\ensuremath{{^{#1}\mathrm{B}}}}
\newcommand{\C}[1]{\ensuremath{{^{#1}\mathrm{C}}}}
\newcommand{\N}[1]{\ensuremath{{^{#1}\mathrm{N}}}}
\renewcommand{\O}[1]{\ensuremath{{^{#1}\mathrm{O}}}}
\newcommand{\F}[1]{\ensuremath{{^{#1}\mathrm{F}}}}
\newcommand{\Na}[1]{\ensuremath{{^{#1}\mathrm{Na}}}}
\newcommand{\Ne}[1]{\ensuremath{{^{#1}\mathrm{Ne}}}}
\newcommand{\Mg}[1]{\ensuremath{{^{#1}\mathrm{Mg}}}}
\newcommand{\Al}[1]{\ensuremath{{^{#1}\mathrm{\!Al}}}}
\newcommand{\Si}[1]{\ensuremath{{^{#1}\mathrm{Si}}}}
\newcommand{\Fe}[1]{\ensuremath{{^{#1}\mathrm{Fe}}}}
\newcommand{\el}[2]{\ensuremath{{^{#2}\mathrm{#1}}}}
\newcommand{\avg}[1]{\ensuremath{\langle#1\rangle}}
\newcommand{\rate}[1]{\ensuremath{r_\mathrm{#1}}}
\newcommand{\sv}[1]{\ensuremath{\langle\sigma v\rangle_{#1}}}
\newcommand{\svpp}{\ensuremath{\langle\sigma v\rangle_\mathrm{pp}}}
\newcommand{\svpD}{\ensuremath{\langle\sigma v\rangle_\mathrm{pD}}}
\newcommand{\life}[2]{\ensuremath{\tau_\mathrm{#1}(\mathrm{#2})}}
\newcommand{\fpp}[1]{\ensuremath{f_\mathrm{pp#1}}}


\begin{document}

\hints{6}

\begin{enumerate}


\item {\bf Conceptual questions: Gamow peak} 

  \begin{enumerate}
  \item $e^{-b/E^{1/2}}$ is the energy dependence of the tunnelling
    probability; $e^{-E/kT}$ is the tail of the Maxwell-Boltzmann
    distribution.

  \item See Figure 6.4.

  \item See Figure 6.4.

  \item The Coulomb barrier is lower for hydrogen burning (smaller $Z$ and
    smaller $A$).

  \item Hint: the neutron has no charge.

\end{enumerate}


\item{\bf Hydrogen burning}

  See Table~6.1 for the atomic masses of the nuclei involved.
  \begin{enumerate}
  \item $m_\mathrm{u} c^2 = 931.49$\,MeV. \\
    Energy release per reaction in each step of the pp1 chain: 1.44,
    5.49 and 12.85\,MeV.

  \item $2\times(1.44 - 0.263) + 2\times 5.49 + 12.85 = 26.21$\,MeV.

  \item $26.21$\,MeV$/(4m_{\rm H}) = 3.91\times10^{24}$\,MeV/g $=
    6.3\times10^{18}$\,erg/gram H.

  \end{enumerate}


%(a) Calculate the energy released per reaction in MeV (the $Q$-value) for the
%three reactions in the pp1 chain. (Hint: first calculate the equivalent
%of $m_\mathrm{u} c^2$ in MeV.)  \\
%(b) What is the total effective $Q$-value for the conversion of four 
%H nuclei into \He{4} by the pp1 chain? Note that in the first reaction
%($\H{1} + \H{1} \rightarrow \H{2} + \e^+ + \nu$) a neutrino is
%released with (on average) an energy of 0.263 MeV.\\
%(c) Calculate the energy released by the pp1-chain in ergs/gram. \\
%(b) Will the answer you get in (c) be different for the pp2 chain, the 
%pp3 chain or the CNO cycle? If so, why? If not, why not?

\item {\bf Relative abundances for CN equilibrium}

  $X_{\N{15}} : X_{\C{13}} : X_{\C{12}} : X_{\N{14}} = \tau(\N{15}) :
  \tau(\C{13}) : \tau(\C{12}) : \tau(\N{14}) $

%Estimate the
%CN-equilibrium relative abundances if their lifetimes before proton
%capture at $T=2\times10^7$ are:
%\\
%- $\tau (^{15}N)\approx30{\rm~yr}$.\\
%- $\tau (^{13}C)\approx1600{\rm~yr}$.\\
%- $\tau (^{12}C)\approx6600{\rm~yr}$.\\
%- $\tau (^{14}N)\approx6\times10^5{\rm~yr}$.


\item {\bf Helium burning}

  To produce \C{12} you need 3 $\alpha$'s: 7.28 MeV per reaction $=
  3.65 \times 10^{23}$\,MeV/gram \C{12}.

  To procuce \O{16} you need 4 $\alpha$'s:

  \begin{tabular}{rclcl}
    $3 \alpha$ &$\rightarrow$&$ ^{12}\rm{C}$ && $7.28\,\rm{MeV}$\\
    $ \alpha +  \C{12}$&$\rightarrow$&$ ^{16}\rm{O}$ && $7.16\,\rm{MeV}$\\
    \hline
    $4 \alpha $&$\rightarrow$ & $ ^{16}\rm{O}$ && $14.44\,\rm{MeV} 
    = 5.44 \times 10^{23}$\,MeV/gram \O{16}
  \end{tabular}
  
  Assume equal mass fractions of \C{12} and \O{16}, so that
  \begin{eqnarray}
    \frac{1}{2} (3.65\times10^{23} + 5.44\times10^{23}) 
    &=& 4.55\times10^{23} \rm{MeV/g}\nonumber \\
    &=& 7.29\times10^{17} \rm{erg/g}\nonumber \\
    &\approx& \frac{1}{10} \text{~ of that for H burning} \nonumber
  \end{eqnarray}
  

\item {\bf Comparing radiative and convective cores}

  \emph{Hand-in exercise.}


\end{enumerate}


\end{document}




\item {\bf Comparing radiative and convective cores}

%%% Answers to question 1 (inlever opgave)

a) $dL_m/dm = \epsilon_{\rm nuc} \Rightarrow L_m =C + \int
	\epsilon_{\rm nuc}dm$, from $L_m(0) = 0 \Rightarrow C=0$. 
\[
L_r = 
\begin{cases}
\epsilon_c\left(m - \frac{(m)^2}{0.2 M_*}\right) & \text{if $m/M_*<0.1$}\\
\frac{0.1 \epsilon_c M_*}{2}& \text{if $0.1 <m/M_* <1$}\\
\end{cases}
\]
From  
\[
L_m(M_*) = L_* \Rightarrow \epsilon_c = \frac{2}{0.1}\frac{80 \Lsun}{3 \Msun}
\].  
%Calculate and draw the luminosity profile, $l$, as a function of the
%mass, $m$. Express $\epsilon_c$ in terms of the known quantities for the
%star.\\
\\ b) The hydrogen abundance at every point $m$ in the star ($m <
0.1M_*$) will drop linearly with time from $0.7$ to $0$ in a time
$t_{\rm exh}(m)$ depending on $m$ via $\epsilon_{\rm nuc}(m)$: $t_{\rm
  exh}(m) = Q/\epsilon_{\rm nuc}(m)$.
\[
X(m,t) = 0.7\left(1 - \frac{t}{ Q/\epsilon_{\rm nuc}(m)}\right).
\]
X(0, 100Myr) = 0.33.\\

  c) The hydrogen abundance profile becomes a step function. The amount
of mass that can be used as fuel is 4 times larger (explain why!). The
luminosity ( the rate at which fuel is burned) has not changed. The
lifetime is therefor 4 times longer hwen the core is convective.



%%%%%%%%%%%%%%%%%%%%%%%%%%%%%%%%%%%%%%%%%%%%%%%%%%%%%%%%%%END

\item {\bf Gamow peak} The peak in $<\sigma\upsilon>$ occurs when
$-E/kT-\bar\eta/E^{1/2}$ is at minimum. Take the derivative to find an
expression for $E_0$ and compare with\footnote{Intermediate answer:
$E_0 = 5.665 {\rm keV} (Z_1^2 Z_2^2 m)^{1/3}T_7^{2/3}$}.

The coulomb energy $E_{\rm coul} = \frac{Z_1Z_2e^2}{r_0}$, where $r_0
\approx A^{1/3} 1.44 . 10^{-13}$cm is of the order $E_{\rm coul}(r_0)
\approx Z_1 Z_2$MeV.


%Derive the energy $E_0$ of the {\it Gamow peak}
%as a function of the temperature, assuming that the reaction rate is
%proportional to
%\begin{equation}
%<\sigma\upsilon>=\left(\frac{8}{m\pi}\right)^{1/2}\frac{S_0}{(kT)^{3/2}}
%\int_0^{\infty}e^{-E/kT-\bar\eta/E^{1/2}}dE,
%\end{equation}
%where the factor $\bar\eta=\pi(2m)^{1/2}\frac{Z_1Z_2e^2}{\hbar}$ and
%$m=\frac{m_1m_2}{m_1+m_2}$ is the reduced mass. Calculate $E_0$ for the
%reactions in the {\it pp1} chain, at temperature $T=10^7$ K, and compare
%with the corresponding Coulomb barriers.
%\footnote{Intermediate answer: $E_0 = 5.665 {\rm keV} (Z_1^2 Z_2^2
%m)^{1/3}T_7^{2/3}$}

\item {\bf Temperature sensitivity} Approximate
$<\sigma\upsilon>=C\tau^2e^{-\tau}$, with $\tau=3E_0/kT$.  Calculate
the sensitivity of the reaction rate to the temperature, expressed as
$\nu(T)=\frac{\partial \ln<\sigma\upsilon>}{\partial \ln T}$, for the
{\it pp1} chain and for the $^{14}N, p$ reaction.\footnote{Intermediate answer: $\nu = (\tau -2)/3 = 6.547 (Z_1^2 Z_2^2
m)^{1/3}T_7^{1/3} - 2/3 $}



\end{enumerate}

\end{document}
