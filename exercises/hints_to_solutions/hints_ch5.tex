\documentclass[11pt,a4paper]{report}

\usepackage{epsfig}
\usepackage{amsmath}
\usepackage{amssymb}
\usepackage{txfonts}

\textwidth 15.5cm
\textheight 23.0cm
\oddsidemargin 0cm
\evensidemargin 0cm
\topmargin 0cm

\pagestyle{empty}

\newcommand{\hints}[1]{\centerline{{\large\bf 
      Stellar Evolution -- Hints to exercises -- Chapter #1}} \bigskip
\renewcommand{\labelenumi}{#1.\arabic{enumi}}}

% for making counters of style 1.1, 1.2, etc
%\renewcommand{\labelenumi}{1.\arabic{enumi}}

% symbol definitions:
\newcommand{\Msun}{\ensuremath{{M}_\odot}}
\newcommand{\Lsun}{\ensuremath{{L}_\odot}}
\newcommand{\Rsun}{\ensuremath{{R}_\odot}}
\newcommand{\Mcore}{\ensuremath{M_\mathrm{c}}}
\newcommand{\MCh}{\ensuremath{M_\mathrm{Ch}}}
\newcommand{\Teff}{\ensuremath{T_\mathrm{eff}}}
\newcommand{\lum}{\ensuremath{l}}
\newcommand{\ledd}{\ensuremath{l_\mathrm{Edd}}}
\newcommand{\Ledd}{\ensuremath{L_\mathrm{Edd}}}
\newcommand{\Prad}{\ensuremath{P_\mathrm{rad}}}
\newcommand{\Pgas}{\ensuremath{P_\mathrm{gas}}}
\newcommand{\Pion}{\ensuremath{P_\mathrm{ion}}}
\newcommand{\Prot}{\ensuremath{P_\mathrm{rot}}}
\newcommand{\GammaC}{\ensuremath{{\Gamma_\mathrm{C}}}}
\newcommand{\gammad}{\ensuremath{{\gamma_\mathrm{ad}}}}
\newcommand{\gradad}{\ensuremath{{\nabla_\mathrm{ad}}}}
\newcommand{\gradrad}{\ensuremath{{\nabla_\mathrm{rad}}}}
\newcommand{\tdyn}{\ensuremath{{\tau_\mathrm{dyn}}}}
\newcommand{\tKH}{\ensuremath{{\tau_\mathrm{KH}}}}
\newcommand{\tnuc}{\ensuremath{{\tau_\mathrm{nuc}}}}
\newcommand{\tms}{\ensuremath{{\tau_\mathrm{MS}}}}
\newcommand{\enuc}{\ensuremath{{\epsilon_\mathrm{nuc}}}}
\newcommand{\eneu}{\ensuremath{{\epsilon_\nu}}}
\newcommand{\egrav}{\ensuremath{{\epsilon_\mathrm{gr}}}}
\newcommand{\cP}{\ensuremath{{c_{\scriptstyle P}}}}
\newcommand{\cV}{\ensuremath{{c_{\scriptstyle V}}}}
\newcommand{\lmix}{\ensuremath{{\ell_\mathrm{m}}}}
\newcommand{\HP}{\ensuremath{{H_{\scriptstyle P}}}}
\newcommand{\vel}{\ensuremath{{\upsilon}}}
\newcommand{\vconv}{\ensuremath{{\vel_\mathrm{c}}}}
\newcommand{\vsound}{\ensuremath{{\vel_\mathrm{s}}}}
\newcommand{\veq}{\ensuremath{{\vel_\mathrm{eq}}}}
\newcommand{\vect}[1]{\mbox{\boldmath $#1$}}
\newcommand{\gradient}{\mbox{\boldmath $\nabla$}}

\newcommand{\Chi}{\ensuremath{\raisebox{0.4ex}{$\chi$}}}

\newcommand{\kmsMpc}{\ensuremath{%
  \mathrm{km}\,\mathrm{s}^{-1}\,\mathrm{Mpc}^{-1}}}

\newcommand{\arrow}{\ensuremath{\Rightarrow}}
\newcommand{\lsim}{\mathrel{\hbox{\rlap{\lower.55ex \hbox {$\sim$}}
 \kern-.3em \raise.4ex \hbox{$<$}}}}
\newcommand{\gsim}{\mathrel{\hbox{\rlap{\lower.55ex \hbox {$\sim$}}
 \kern-.3em \raise.4ex \hbox{$>$}}}}

\newcommand{\beq}{\begin{equation}}
\newcommand{\eeq}{\end{equation}}
\newcommand{\beqar}{\begin{eqnarray}}
\newcommand{\eeqar}{\end{eqnarray}}
\newcommand{\der}{\ensuremath{\mathrm{d}}}
\newcommand{\dd}{\partial}

\newcommand{\half}{\ensuremath{{\textstyle\frac{1}{2}}}}
\newcommand{\third}{\ensuremath{{\textstyle\frac{1}{3}}}}
\newcommand{\onesixth}{\ensuremath{{\textstyle\frac{1}{6}}}}

\newcommand{\amu}{\ensuremath{m_\mathrm{u}}}
\newcommand{\e}{\ensuremath{\mathrm{e}}}
\newcommand{\n}{\ensuremath{\mathrm{n}}}
\newcommand{\p}{\ensuremath{\mathrm{p}}}
\renewcommand{\H}[1]{\ensuremath{{^{#1}\mathrm{H}}}}
\newcommand{\D}{\ensuremath{\mathrm{D}}}
\newcommand{\He}[1]{\ensuremath{{^{#1}\mathrm{He}}}}
\newcommand{\Li}[1]{\ensuremath{{^{#1}\mathrm{Li}}}}
\newcommand{\Be}[1]{\ensuremath{{^{#1}\mathrm{Be}}}}
\newcommand{\B}[1]{\ensuremath{{^{#1}\mathrm{B}}}}
\newcommand{\C}[1]{\ensuremath{{^{#1}\mathrm{C}}}}
\newcommand{\N}[1]{\ensuremath{{^{#1}\mathrm{N}}}}
\renewcommand{\O}[1]{\ensuremath{{^{#1}\mathrm{O}}}}
\newcommand{\F}[1]{\ensuremath{{^{#1}\mathrm{F}}}}
\newcommand{\Na}[1]{\ensuremath{{^{#1}\mathrm{Na}}}}
\newcommand{\Ne}[1]{\ensuremath{{^{#1}\mathrm{Ne}}}}
\newcommand{\Mg}[1]{\ensuremath{{^{#1}\mathrm{Mg}}}}
\newcommand{\Al}[1]{\ensuremath{{^{#1}\mathrm{\!Al}}}}
\newcommand{\Si}[1]{\ensuremath{{^{#1}\mathrm{Si}}}}
\newcommand{\Fe}[1]{\ensuremath{{^{#1}\mathrm{Fe}}}}
\newcommand{\el}[2]{\ensuremath{{^{#2}\mathrm{#1}}}}
\newcommand{\avg}[1]{\ensuremath{\langle#1\rangle}}
\newcommand{\rate}[1]{\ensuremath{r_\mathrm{#1}}}
\newcommand{\sv}[1]{\ensuremath{\langle\sigma v\rangle_{#1}}}
\newcommand{\svpp}{\ensuremath{\langle\sigma v\rangle_\mathrm{pp}}}
\newcommand{\svpD}{\ensuremath{\langle\sigma v\rangle_\mathrm{pD}}}
\newcommand{\life}[2]{\ensuremath{\tau_\mathrm{#1}(\mathrm{#2})}}
\newcommand{\fpp}[1]{\ensuremath{f_\mathrm{pp#1}}}


\begin{document}

\hints{5}

\begin{enumerate}

\item {\bf Radiation transport} 
  \begin{enumerate}

%How long does it take a typical photon to travel from the
%    center of the Sun to the surface? How does this relate to the
%    thermal timescale of the Sun? 

  \item 

    The mean free path of a photon is $\ell_\mathrm{ph} = 1/(\kappa \rho)$,
    where $\kappa$ is the opacity coefficient. The photons escape in a random
    walk fashion, so the distance $d$ travelled after $N$ scatterings is $d^2
    \approx N \ell_{\rm ph}^2$. In order to travel from the center to the
    surface of the Sun, $N \approx (R_{\odot}/\ell_{\rm ph})^2$ steps are
    needed. 

    To obtain an estimate of the time needed for this random-walk
    process, take `typical' values $\kappa \approx
    1\,\rm{cm}^{2}/\rm{g}$ and $\rho \approx \bar{\rho}_\odot = 1.4$
    g\,cm$^{-3}$, so that $\ell_{ph} \approx 1 \rm{cm}$. The time $t$
    needed is $t = N \ell_{\rm ph}/c \approx R_{\odot}^2/(c \ell_{\rm
      ph}) \approx 1.5\times 10^{11} \rm{s} \approx 10^4 \rm{y}$.

    (N.B. The actual time is much longer, because $\kappa \gg 1
    \rm{cm}^2/\rm{g}$ in the outer layers, while $\rho > \bar{\rho}$
    in the centre. Therefore the average $\ell_\mathrm{ph}$ is much
    shorter than 1~cm, and $t \approx 10^7 \rm{yrs} \approx \tau_{\rm
      KH}$).
 
  \item $\ell_{ph} \approx 1 \rm{cm}$ and $dT/dr \approx
    (T_{\rm eff} - T_c)/R_{\odot} \approx 10^{-4}$\,K/cm.
    
%Estimate a typical value for $\frac{dT}{dr}$ to
%    show that the difference in temperature $\Delta T$ between to surfaces
%    in the solar interior one photon mean free path $l_{ph}$ apart is
    
  \item The total flux emitted in all directions by a blackbody is $F
    = \sigma T^4$, where $T \approx 10^7 K$ and $\sigma = 5.67\times
    10^{-5}$~erg/K$^4$cm$^2$s.

  \item $F \propto T^4 \Rightarrow \displaystyle \frac{\Delta F}{F}
    \propto 4\frac{\Delta T}{T}$, where $\Delta T/T \approx 10^{-11}$.
  
    The net outward flux $\Delta F = 4 F \Delta T/T \approx
    4\sigma\, (10^7)^4 \times 10^{-11}$~erg/cm$^2$s.  The
    corresponding luminosity comes from multiplying by the surface of
    sphere with radius $r\sim\Rsun/10$: $L = 4 \pi r^2\, \Delta F
    \approx 1.4 \times 10^{34}$~erg/s. This is within one order of
    magnitude of the solar surface luminosity.

%Show that the minute deviation from isotropy between to
%    surfaces in the solar interior one photon mean free path apart at
%    $r\sim\Rsun/10$ and $T\sim10^7$ K, is sufficient for the transfer
%    of energy that results in the luminosity of the sun. 
    
  \item The mean free path for photons becomes very long near the
    surface, where they escape (we can see them on Earth). Then LTE is
    no longer valid.

%Why does the diffusion approximation for radiation transport
%    break down when the surface (photosphere) of a star is approached?

  \end{enumerate}


\item {\bf Opacity}

  \begin{enumerate}

  \item Correlate Fig.~5.2 with Section~5.3.1. Electron scattering
    gives constant $\kappa$ at high $T$ and low $\rho$. Free-free and
    bound-free absorption give $\kappa \propto T^{-3.5}$ between
    $10^4$\,K and $\sim10^7$\,K, depending on $\rho$. The H$^{-}$ ion
    gives $\kappa \propto T^9$ between 3000\,K and $10^4$\,K. Molecules
    are important for $T<4000$\,K and dust absorption dominates at
    $T<1500$\,K.

  \item Draw the relations (5.30), (5.32), (5.33) and (5.34) in
    Fig.5.2a, putting in the right values for $X$, $Z$ and $\rho$.

  \item Use $\ell_{\rm ph} = 1/(\kappa\rho)$ and estimate $\log\rho$
    and $\log\kappa$ from the location of the 1\,\Msun\ model in
    Fig.~5.2b: \\[1ex]
    \begin{tabular}{llll}
    $T = 10^7$ K: & $\log\rho \approx 2.0$ & $\log\kappa \approx 0.5$
      & $\ell \approx 3\times 10^{-2}$\,cm \\
    $T = 10^5$ K: & $\log\rho \approx -2.5$ & $\log\kappa \approx 4.5$
      & $\ell \approx 10^{-2}$\,cm \\
    $T = 10^4$ K: & $\log\rho \approx -6.0$& $\log\kappa \approx 2.0$
      & $\ell \approx 10^4$\,cm \\
    \end{tabular}

  \item Take the derivative of eq.~(5.21) to obtain $\partial
    U_\nu/\partial T$ and divide by $\kappa_\nu = \kappa_0
    \nu^{-\alpha}$. This quantity should be integrated over $\nu$
    according to eq.~(5.24). Hint: substitute $x = h\nu/kT$.

  \end{enumerate}

\item {\bf Mass-luminosity relation for stars in radiative equilibrium}

  \emph{Hand-in exercise.}


\item {\bf Conceptual questions about convection} 

  \emph{Hand-in exercise.}


\item {\bf Applying Schwarzschild's criterion} 

  Assume $\nabla_{\rm ad} =\nabla_{\rm ad,ideal}= 0.4$. If
  $\nabla_{\rm rad} > 0.4$ energy transport is by convection,
  otherwise by radiation.
  \[
  \nabla_{\rm rad} = \frac{3}{16 \pi a c G} \frac{\kappa \lum P}{m
    T^4}
  \]

  \begin{enumerate}
  \item Low mass stars have low temperatures and therefore high
    opacities, leading to $\nabla_{\rm rad} > 0.4$ everywhere.
  \item Use the ideal gas law to calculate $P$ and use the numbers
    from the table.
  \end{enumerate}



\item{\bf Eddington Luminosity}

\begin{enumerate}
%  The Eddington luminosity is the maximum luminosity a star (with
%  radiative energy transport) can have, where radiation force equals
%  gravity.

  \item 
    See Section~5.4.
    
    The outward acceleration $a_{\rm rad}$ due to the radiation
    pressure gradient is
    \[
    a_{\rm rad} = \frac{1}{\rho}\,\frac{dP_{\rm rad}}{dr} = =
    \frac{4aT^3}{3\rho}\frac{dT}{dr}
    \]
    which cannot be larger than the gravitational acceleration
    $Gm^2/r$. Assuming radiation trasnport, $dT/dr$ is given by eq.~(5.16),
    which after substitution gives 
    \[
    \lum < \frac{4 \pi c G m}{\kappa}.
    \]

%Show that
%\[
%L_{r, \max} = \frac{4 \pi c G m}{\kappa}.
%\]

  \item

    Starting with $L/L_{\rm Edd}$, use the equation for the radiative
    temperature gradient to substitute $L$, then use $4 a T^3 dT/dr =
    dP_{\rm rad}/dr$, then use $P_{\rm rad} = P(1-\beta)$ and get rid
    of $dP/dr$ using the HE equation.

%Consider a star with a uniform opacity $\kappa$ and of uniform
%  parameter $\beta = P_{\rm rad}/P$. Show that $L/L_{\rm Edd}=1-\beta$ for
%  such a star.

\item 

  Rewrite \gradrad\ as
  \[
  \nabla_{\rm rad} = \frac{\kappa \lum}{4 \pi c G m}
  \frac{P}{\frac{4}{3}a T^4 }
  \]
  Now simplify the expression by introducing $\lum_{\rm Edd}$ and $P_{\rm
    rad}$. Then substitute into the Schwarz\-schild criterion.

%Show that the Schwarzschild criterium for stability against
%  convection $(\nabla_{\rm rad} < \nabla_{\rm ad})$ can be rewritten
%  as:
%\[
%\frac{L_r}{L_{\rm Edd}} = 4 \frac{P_{\rm rad}}{P} \nabla_{\rm ad}
%\]

\item On the boundary of the core there is a transition from
  convective to radiative energy transport. So, on the boundary the
  Schwarzschild criterion can be written as an equality:
  \[
  \frac{\lum_\mathrm{core}}{\lum_{\rm Edd, core}} = 4 (1-\beta)\nabla_{\rm
    ad}.
  \]
  Outside the core there is no energy generation, so $\lum_{\rm core}
  = L$. From (b) you know that $L = (1-\beta) L_{\rm Edd}$.  From (a)
  you know that $\lum_{\rm Edd, core} \propto M_{\rm core}$ and also
  $L_{\rm Edd} \propto M$. Now combine all this knowledge.

  \[
  \begin{array}{lcl}
    \lum_{\rm core} ~ = ~ L &=& (1-\beta)L_{\rm Edd} \\
    \lum_{\rm Edd, core} &=& \displaystyle\frac{M_{\rm core}}{M} L_{\rm Edd} \\
  \end{array} ~ \Bigg\} ~
  \Rightarrow ~ \displaystyle
  \frac{\lum_{\rm core}}{\lum_{\rm Edd, core}} = (1-\beta)\frac{M}{M_{\rm
      core}}
  \]
  \[
  \Rightarrow ~ 4(1-\beta)\nabla_{ad} =
  (1-\beta)\frac{M}{M_{\rm core}}
  \]
  and there you are.

%Consider again the star of question b). By assuming that
%it has a convective core, and no nuclear energy generation outside the core,
%show that the mass fraction of this core is given by
%\[
%\frac{1}{4\nabla_{ad}}.
%\]

    

\end{enumerate}

\end{enumerate}
\end{document}


%% Answers to hand-in exercise:

\item {\bf Mass Luminosity relation}
  \begin{enumerate}

%  \item Derive how the central temperature, $T_c$, scales with the
%    mass, $M$, radius, $R$, and luminosity, $L$, for a star in which the
%    energy transport is by radiation. To do this, use
%    \begin{equation}
%      \frac{dT}{dr}=-\frac{3}{16\pi ac}\frac{\kappa\rho l}{T^3 r^2}
%    \end{equation}
%    at $r=R/2$ and assume the temperature is proportional to $T_c$, taking
%    $l\sim L$ and estimating $\frac{dT}{dr}\sim-\frac{T_c}{R}$.
    
  \item Hint: use $\rho \sim M/R^3$
%Derive how $T_c$ scales with $M$ and $R$, using the
%    hydrostatic equilibrium equation, and assuming that the ideal gas law holds.
  \item Hint: use $P \sim \rho T / \mu$
  \item $L \propto \mu^4 M^3$
%Combine the results obtained in (a) and (b), to derive how $L$
%    scales with $M$ and $R$ for a star whose energy transport is
%    radiative. 

  \item The expression for $\frac{dT}{dr}$ used in a) is only valid
  for radiative stars while low mass stars are fully
  convective. Further you probably assumed an ideal gas law, which
  breaks down for the high mass stars where the radiation pressure
  becomes important. You also had to assume $\kappa$ constant which is
  in reality a function of density and temperature.


%Compare your answer to the relation between $M$ and $L$ which
%    you derived from observations. Why does the derived powerlaw
%    relation starts to deviate from observations for low mass stars?
%    Why does it deviate for high mass stars?
    
  \end{enumerate}


\item {\bf Conceptual Questions} 

  \begin{itemize}
  \item See Section 4.5.2.
  \item See Figure 4.3.
  \item See Section 4.5.2.
  \end{itemize}  


\item {\bf Comparing radiative and convective cores}

%%% Answers to question 1 (inlever opgave)

a) $dL_m/dm = \epsilon_{\rm nuc} \Rightarrow L_m =C + \int
	\epsilon_{\rm nuc}dm$, from $L_m(0) = 0 \Rightarrow C=0$. 
\[
L_r = 
\begin{cases}
\epsilon_c\left(m - \frac{(m)^2}{0.2 M_*}\right) & \text{if $m/M_*<0.1$}\\
\frac{0.1 \epsilon_c M_*}{2}& \text{if $0.1 <m/M_* <1$}\\
\end{cases}
\]
From  
\[
L_m(M_*) = L_* \Rightarrow \epsilon_c = \frac{2}{0.1}\frac{80 \Lsun}{3 \Msun}
\].  
%Calculate and draw the luminosity profile, $l$, as a function of the
%mass, $m$. Express $\epsilon_c$ in terms of the known quantities for the
%star.\\
\\ b) The hydrogen abundance at every point $m$ in the star ($m <
0.1M_*$) will drop linearly with time from $0.7$ to $0$ in a time
$t_{\rm exh}(m)$ depending on $m$ via $\epsilon_{\rm nuc}(m)$: $t_{\rm
  exh}(m) = Q/\epsilon_{\rm nuc}(m)$.
\[
X(m,t) = 0.7\left(1 - \frac{t}{ Q/\epsilon_{\rm nuc}(m)}\right).
\]
X(0, 100Myr) = 0.33.\\

  c) The hydrogen abundance profile becomes a step function. The amount
of mass that can be used as fuel is 4 times larger (explain why!). The
luminosity ( the rate at which fuel is burned) has not changed. The
lifetime is therefor 4 times longer hwen the core is convective.


\item {\bf Eddington Standard Model}
%In order to model the Sun and more massive stars, the so-called
%Standard Model of A. S. Eddington is used. Eddington derived stellar
%models in which the radiation pressure is a constant fraction of the
%total pressure, in other words: $\beta = P_\mathrm{gas}/P$ and $1-\beta=
%P_\mathrm{rad}/P$ are constant throughout the star.
\begin{enumerate}
\item 

The virial theorem in its general form is given by 
\[
E_{\rm gr} = - 3 \int P dV
\]
For an ideal gas $P_{\rm gas} = nkT$ and  the energy density
$\frac{3}{2}nkT =\frac{3}{2}P_{\rm gas} =\frac{3}{2}\beta P$. For
radiation $P_{\rm rad} = \frac{1}{3}aT^4 $ and  the energy density
$aT^4 = 3 P_{\rm rad} = 3 (1-\beta) P$.

So the total energy density is given by $ U = \frac{3}{2}\beta P + 3
(1-\beta) P = (3 - \frac{3}{2}\beta) P$. Use this to rewrite $P$ in
the virial theorem.
\[
E_{\rm gr} = - \frac{2}{2-\beta} \int uU dV = - \frac{2}{2-\beta} E
_{\rm int}.
\]
Now derive the expression for the total energy yourself.



%Show that the virial theorem leads to
%\[
%E_{tot}=\frac{\beta}{2}E_{gr}=-\frac{\beta}{2-\beta}E_{in},
%\]
%for a classical, non-relativistic gas. What happens in the limits
%$\beta\rightarrow 1$ and $\beta\rightarrow 0$?

\item 
Using the radiation pressure write the temperature as a function of $P$
and $\beta$.
\[
P_{\rm rad} = (1-\beta)P = aT^4 \Rightarrow T = T(\beta, P).
\]
Then insert this in the ideal gas law and you are left with an equation
independent of T. Rewrite and find that 
\[
P \propto \rho^{4/3} \Rightarrow n =3
\]


%Show that the Eddington assumption leads to an equation of state
%that can be written in polytropic form, with a polytropic index $n=3$.

\item 

%Also show that the corresponding constant $K$ is not fixed but depends
%on $\beta$ and the mean molecular weight $\mu$ as
%\[
%K=\frac{2.67 \times
%10^{15}}{\mu^{4/3}}\left(\frac{1-\beta}{\beta^4}\right)^{1/3}.
%\]

\item 
%Show that for this $n$ the mass is uniquely determined by the value of $K$.
(Use results of problem P.2.b).

\item 
%Now use the results from above to derive the relation $M=M(\beta,\mu)$.
%This is useful for numerically solving the amount of radiation pressure
%for a star with a given mass.

\end{enumerate}
%\item Find the connection between $c_p$ and $c_v$ for an ideal gas.

%\item Neglecting the radiation pressure and assumming $\bar m$ is 0.7 amu,
%determine if the energy transport is convective or radiative for the following
%values:\\$r=0.1\Rsun;~ m=0.028\Msun;~ l=24.2\Lsun;~ T=2.2\times10^7{\rm~K};~
%\rho=3.1\times10^{4}{\rm~kg~m^{-3}}$ and $\kappa=4\times10^{-2}
%{\rm~m^2~kg{-1}}$.


\end{enumerate}

\end{document}


\pagebreak

  \item Compare two stars with the same luminosity but a different
    interior opacity, e.g. because the composition is different. Which
    star has the largest temperature gradient?

\item How does the radiative temperature gradient scale with pressure for stars
whose temperature is scaled by $T\sim P^{\frac{1}{1+n}}$? Assume energy
generation is constant through the star (i.e., $l\sim m$) and that the
opacity has
a power law dependence of temperature and pressure, $\kappa\sim P^aT^b$.
Generally, in the interiors, $a\sim1$ and $b\sim -4.5$, for high temperatures.
Use n=1.5 and n=3 and indicate to what situation the results correspond to.

\item Neglecting the radiation pressure and assuming that the average
particle has a mass $\bar m=0.7$ amu,
determine if the energy transport is convective or radiative for the following
values at different radii in a model for a $5\Msun$ main sequence star and
check if they are what is expected for such a star:

%\begin{table}
\begin{center}
\begin{tabular}{|c|c|c|c|c|c|}
  \hline
  {$r/\Rsun$}&{$m/\Msun$}&{$L_r/\Lsun$}&{$T{\rm~[K]}$}&
  {$\rho{\rm~[g~cm^{-3}]}$}&{$\kappa{\rm~[g^{-1}~cm^{2}]}$}\\
  \hline
  0.242&0.199&$3.40\times10^{2}$&$2.52\times10^{7}$&18.77&0.435\\
  0.670&2.487&$5.28\times10^{2}$&$1.45\times10^{7}$&6.91&0.585\\
  \hline
\end{tabular}
\end{center}
~\\


\pagebreak

