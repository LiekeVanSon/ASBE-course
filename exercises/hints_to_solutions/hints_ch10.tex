\documentclass[11pt,a4paper,fleqn]{report}

\usepackage{epsfig}
\usepackage{amsmath}
\usepackage{amssymb}
\usepackage{txfonts}

\textwidth 15.5cm
\textheight 23.0cm
\oddsidemargin 0cm
\evensidemargin 0cm
\topmargin 0cm

\pagestyle{empty}

\newcommand{\hints}[1]{\centerline{{\large\bf 
      Stellar Evolution -- Hints to exercises -- Chapter #1}} \bigskip
\renewcommand{\labelenumi}{#1.\arabic{enumi}}}

% for making counters of style 1.1, 1.2, etc
%\renewcommand{\labelenumi}{1.\arabic{enumi}}

% symbol definitions:
\newcommand{\Msun}{\ensuremath{{M}_\odot}}
\newcommand{\Lsun}{\ensuremath{{L}_\odot}}
\newcommand{\Rsun}{\ensuremath{{R}_\odot}}
\newcommand{\Mcore}{\ensuremath{M_\mathrm{c}}}
\newcommand{\MCh}{\ensuremath{M_\mathrm{Ch}}}
\newcommand{\Teff}{\ensuremath{T_\mathrm{eff}}}
\newcommand{\lum}{\ensuremath{l}}
\newcommand{\ledd}{\ensuremath{l_\mathrm{Edd}}}
\newcommand{\Ledd}{\ensuremath{L_\mathrm{Edd}}}
\newcommand{\Prad}{\ensuremath{P_\mathrm{rad}}}
\newcommand{\Pgas}{\ensuremath{P_\mathrm{gas}}}
\newcommand{\Pion}{\ensuremath{P_\mathrm{ion}}}
\newcommand{\Prot}{\ensuremath{P_\mathrm{rot}}}
\newcommand{\GammaC}{\ensuremath{{\Gamma_\mathrm{C}}}}
\newcommand{\gammad}{\ensuremath{{\gamma_\mathrm{ad}}}}
\newcommand{\gradad}{\ensuremath{{\nabla_\mathrm{ad}}}}
\newcommand{\gradrad}{\ensuremath{{\nabla_\mathrm{rad}}}}
\newcommand{\tdyn}{\ensuremath{{\tau_\mathrm{dyn}}}}
\newcommand{\tKH}{\ensuremath{{\tau_\mathrm{KH}}}}
\newcommand{\tnuc}{\ensuremath{{\tau_\mathrm{nuc}}}}
\newcommand{\tms}{\ensuremath{{\tau_\mathrm{MS}}}}
\newcommand{\enuc}{\ensuremath{{\epsilon_\mathrm{nuc}}}}
\newcommand{\eneu}{\ensuremath{{\epsilon_\nu}}}
\newcommand{\egrav}{\ensuremath{{\epsilon_\mathrm{gr}}}}
\newcommand{\cP}{\ensuremath{{c_{\scriptstyle P}}}}
\newcommand{\cV}{\ensuremath{{c_{\scriptstyle V}}}}
\newcommand{\lmix}{\ensuremath{{\ell_\mathrm{m}}}}
\newcommand{\HP}{\ensuremath{{H_{\scriptstyle P}}}}
\newcommand{\vel}{\ensuremath{{\upsilon}}}
\newcommand{\vconv}{\ensuremath{{\vel_\mathrm{c}}}}
\newcommand{\vsound}{\ensuremath{{\vel_\mathrm{s}}}}
\newcommand{\veq}{\ensuremath{{\vel_\mathrm{eq}}}}
\newcommand{\vect}[1]{\mbox{\boldmath $#1$}}
\newcommand{\gradient}{\mbox{\boldmath $\nabla$}}

\newcommand{\Chi}{\ensuremath{\raisebox{0.4ex}{$\chi$}}}

\newcommand{\kmsMpc}{\ensuremath{%
  \mathrm{km}\,\mathrm{s}^{-1}\,\mathrm{Mpc}^{-1}}}

\newcommand{\arrow}{\ensuremath{\Rightarrow}}
\newcommand{\lsim}{\mathrel{\hbox{\rlap{\lower.55ex \hbox {$\sim$}}
 \kern-.3em \raise.4ex \hbox{$<$}}}}
\newcommand{\gsim}{\mathrel{\hbox{\rlap{\lower.55ex \hbox {$\sim$}}
 \kern-.3em \raise.4ex \hbox{$>$}}}}

\newcommand{\beq}{\begin{equation}}
\newcommand{\eeq}{\end{equation}}
\newcommand{\beqar}{\begin{eqnarray}}
\newcommand{\eeqar}{\end{eqnarray}}
\newcommand{\der}{\ensuremath{\mathrm{d}}}
\newcommand{\dd}{\partial}

\newcommand{\half}{\ensuremath{{\textstyle\frac{1}{2}}}}
\newcommand{\third}{\ensuremath{{\textstyle\frac{1}{3}}}}
\newcommand{\onesixth}{\ensuremath{{\textstyle\frac{1}{6}}}}

\newcommand{\amu}{\ensuremath{m_\mathrm{u}}}
\newcommand{\e}{\ensuremath{\mathrm{e}}}
\newcommand{\n}{\ensuremath{\mathrm{n}}}
\newcommand{\p}{\ensuremath{\mathrm{p}}}
\renewcommand{\H}[1]{\ensuremath{{^{#1}\mathrm{H}}}}
\newcommand{\D}{\ensuremath{\mathrm{D}}}
\newcommand{\He}[1]{\ensuremath{{^{#1}\mathrm{He}}}}
\newcommand{\Li}[1]{\ensuremath{{^{#1}\mathrm{Li}}}}
\newcommand{\Be}[1]{\ensuremath{{^{#1}\mathrm{Be}}}}
\newcommand{\B}[1]{\ensuremath{{^{#1}\mathrm{B}}}}
\newcommand{\C}[1]{\ensuremath{{^{#1}\mathrm{C}}}}
\newcommand{\N}[1]{\ensuremath{{^{#1}\mathrm{N}}}}
\renewcommand{\O}[1]{\ensuremath{{^{#1}\mathrm{O}}}}
\newcommand{\F}[1]{\ensuremath{{^{#1}\mathrm{F}}}}
\newcommand{\Na}[1]{\ensuremath{{^{#1}\mathrm{Na}}}}
\newcommand{\Ne}[1]{\ensuremath{{^{#1}\mathrm{Ne}}}}
\newcommand{\Mg}[1]{\ensuremath{{^{#1}\mathrm{Mg}}}}
\newcommand{\Al}[1]{\ensuremath{{^{#1}\mathrm{\!Al}}}}
\newcommand{\Si}[1]{\ensuremath{{^{#1}\mathrm{Si}}}}
\newcommand{\Fe}[1]{\ensuremath{{^{#1}\mathrm{Fe}}}}
\newcommand{\el}[2]{\ensuremath{{^{#2}\mathrm{#1}}}}
\newcommand{\avg}[1]{\ensuremath{\langle#1\rangle}}
\newcommand{\rate}[1]{\ensuremath{r_\mathrm{#1}}}
\newcommand{\sv}[1]{\ensuremath{\langle\sigma v\rangle_{#1}}}
\newcommand{\svpp}{\ensuremath{\langle\sigma v\rangle_\mathrm{pp}}}
\newcommand{\svpD}{\ensuremath{\langle\sigma v\rangle_\mathrm{pD}}}
\newcommand{\life}[2]{\ensuremath{\tau_\mathrm{#1}(\mathrm{#2})}}
\newcommand{\fpp}[1]{\ensuremath{f_\mathrm{pp#1}}}


\begin{document}

\hints{10}

\begin{enumerate}
  
\item {\bf Conceptual questions}
  \begin{enumerate}

  \item See Section 9.3
    
  \item See Section 9.3.1
    
  \item See Section 9.3.4
    
  \item See Section 10.1
    
  \item See Section 10.1

  \item See Section 10.2.1

  \end{enumerate}


\item {\bf Evolution of the abundace profiles}

  \begin{enumerate}
  \item Take into account: \\
    - the central abundances that you can read from the figures, \\
    - the location of the H-burning shell, \\
    - the extent of a convection zone in the core (if present). \\
    (Hint: look at Fig.~9.10 for an earlier evolution phase).

  \item Idem.

  \item They are modified by convective dredge-up at points E (for
    5\,\Msun) and D (for 1\,\Msun).

  \end{enumerate}


\item {\bf Red giant branch stars}
    
  \begin{enumerate}
  \item Use the virial theorem to calculate the total energy $E_{\rm
      tot} = E_{\rm gr} + E_{\rm int} = \half E_{\rm gr}$.

  \item $R \leq (0.45)^2 \Rsun$.
       
  \item Yes.

  \end{enumerate}


\item {\bf Core mass - luminosity relation for RGB stars}

  \begin{enumerate}

  \item The energy release $q$ per gram of H-burning is (see also
    Ch.5)
    \[
    q = \frac{\der E}{\der M_\mathrm{H}} = \frac{26.73 \times 10^6 \times
      1.6\times 10^{-12}\,\mathrm{erg}}{4 \times 1.67 \times 10^{-24}\,
      \mathrm{g}} = 6.4 \times 10^{18} \text{erg/g}
    \]
    The envelope that provides fuel for the shel has a hydrogen mass
    fraction $X_\mathrm{env} \approx 0.7$, so a core growth $\der M_c$
    corresponds to a $\der M_\mathrm{H} = X_\mathrm{env}\, \der M_c$. \\
    $ \Rightarrow \der E = X_\mathrm{env} q\, \der M_c \Rightarrow
    \displaystyle \frac{\der E}{\der t} = L = X_\mathrm{env} q\,
    \frac{\der M_c}{\der t} $.

  \item Combine (a) with eq.~(9.2), ${L}/\Lsun= 2.3\times10^5
    ({{M_c}/{\Msun}})^6$. Rewrite and integrate:
    \[
    \int_{M_c(t_0)}^{M_c(t)} \left( \frac{M_c}{\Msun} \right)^{-6} \der
      \left( \frac{M_c}{\Msun} \right) 
    = \int_{t_0}^{t} \frac{2.3\times 10^5\Lsun}{X_\mathrm{env} Q \Msun}\,
      \der t
    = \int_{t_0}^{t} 9.9\times10^{-14}\, \der t
    \] 
    with $\der t$ in seconds in the last expression. Integrate to obtain
    an expression for $M_c(t)$.

  \item
    \[
    (t - t_0) = 2.02\times 10^{12}\, \mathrm{sec}\, 
    \Bigg[ \Bigg( \frac{0.15 M_i}{\Msun} \Bigg)^{-5} - (0.45)^{-5} \Bigg]
    \] 
    For a 1~\Msun~star: $\tau_{\rm RGB} \approx 8.4 \times 10^8$~yrs,
    $\tau_{\rm RGB}/\tau_{\rm MS} \approx 0.08$.\\
    For a 2~\Msun~star: $\tau_{\rm RGB} \approx 2.3 \times 10^7$~yrs,
    $\tau_{\rm RGB}/\tau_{\rm MS} \approx 0.016$.
  
  \end{enumerate}


\item {\bf Jump in the composition} 

  \begin{enumerate}

  \item A chemical profile like this can be the result of convection
    in the core. High mass stars have convective cores.

  \item The pressure and temperature are continuous functions of the
    mass coordinate.  A jump in composition (i.e. in $\mu$) has to be
    compensated by a jump in density, if we assume that the ideal gas
    law holds. First express $\mu$ in terms of the discontinuous
    variable $X$.
    \[
    \mu^{-1} = X \frac{2}{1} + Y \frac{3}{4} + Z \frac{A/2}{A} = ... =
    \frac{5}{4}X - \frac{1}{4}Z +\frac{3}{4}
    \]
    Now consider the logarithm of the ideal gas $P = \rho k T/(m_u
    \mu) $ law, which you can rewrite as:
    \[
    \ln \rho = \ln \mu + \ln \left( \frac{P m_u}{k T}\right).
    \]
    Now consider the difference $\Delta$ just above and below the jump
    in composition.
    \[
    \Delta \ln \rho =  \Delta \ln \mu +  
    \Delta \ln \left( \frac{P m_u}{k T}\right).
    \]
    The last term is zero, because $P$ and $T$ are continuous and
    $m_u$ and $k$ are constants, so $\Delta \ln \rho = \Delta \ln
    \mu$.

    \[
    \Delta \ln \rho = \frac{\Delta\rho}{\rho} = -\Delta \ln \left(
      \frac{1}{\mu}\right) = - \ln \left(\frac{5\times 0.7 - 0.02
        +3}{5\times 0.1 - 0.02 +3} \right) = -0.622
    \]
    

  \item For {\bf kramers opacity}: $\kappa_{bf}\sim Z(1+X) \rho
    T^{-3.5}$, then
    \[
    \Delta \ln \kappa = \Delta \ln Z + \Delta \ln (1+X) + \Delta \ln
    \rho -3.5 \Delta \ln T.
    \]
    $T$ and $Z$ are continuous, therefore $\Delta \ln \kappa = \Delta
    \ln (1+X) +\Delta \ln \rho $.
      
    For {\bf electron scattering} $\kappa_e=0.2(1+X)$ then $\Delta
    \kappa / \kappa = \ln (1+0.7) - \ln (1+ 0.1)$
      

  \end{enumerate}
  

\end{enumerate}

\end{document}


\item {\bf The most massive star} Use $L/\Lsun = (M/\Msun)^3$ in
combination with the expreesion for the eddington Luminosity.
