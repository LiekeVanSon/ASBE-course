\documentclass[11pt,a4paper]{report}

\usepackage{epsfig}
\usepackage{amsmath}
\usepackage{amssymb}
\usepackage{txfonts}

\textwidth 15.5cm
\textheight 23.0cm
\oddsidemargin 0cm
\evensidemargin 0cm
\topmargin 0cm

\pagestyle{empty}

\newcommand{\hints}[1]{\centerline{{\large\bf 
      Stellar Evolution -- Hints to exercises -- Chapter #1}} \bigskip
\renewcommand{\labelenumi}{#1.\arabic{enumi}}}

% for making counters of style 1.1, 1.2, etc
%\renewcommand{\labelenumi}{1.\arabic{enumi}}

% symbol definitions:
\newcommand{\Msun}{\ensuremath{{M}_\odot}}
\newcommand{\Lsun}{\ensuremath{{L}_\odot}}
\newcommand{\Rsun}{\ensuremath{{R}_\odot}}
\newcommand{\Mcore}{\ensuremath{M_\mathrm{c}}}
\newcommand{\MCh}{\ensuremath{M_\mathrm{Ch}}}
\newcommand{\Teff}{\ensuremath{T_\mathrm{eff}}}
\newcommand{\lum}{\ensuremath{l}}
\newcommand{\ledd}{\ensuremath{l_\mathrm{Edd}}}
\newcommand{\Ledd}{\ensuremath{L_\mathrm{Edd}}}
\newcommand{\Prad}{\ensuremath{P_\mathrm{rad}}}
\newcommand{\Pgas}{\ensuremath{P_\mathrm{gas}}}
\newcommand{\Pion}{\ensuremath{P_\mathrm{ion}}}
\newcommand{\Prot}{\ensuremath{P_\mathrm{rot}}}
\newcommand{\GammaC}{\ensuremath{{\Gamma_\mathrm{C}}}}
\newcommand{\gammad}{\ensuremath{{\gamma_\mathrm{ad}}}}
\newcommand{\gradad}{\ensuremath{{\nabla_\mathrm{ad}}}}
\newcommand{\gradrad}{\ensuremath{{\nabla_\mathrm{rad}}}}
\newcommand{\tdyn}{\ensuremath{{\tau_\mathrm{dyn}}}}
\newcommand{\tKH}{\ensuremath{{\tau_\mathrm{KH}}}}
\newcommand{\tnuc}{\ensuremath{{\tau_\mathrm{nuc}}}}
\newcommand{\tms}{\ensuremath{{\tau_\mathrm{MS}}}}
\newcommand{\enuc}{\ensuremath{{\epsilon_\mathrm{nuc}}}}
\newcommand{\eneu}{\ensuremath{{\epsilon_\nu}}}
\newcommand{\egrav}{\ensuremath{{\epsilon_\mathrm{gr}}}}
\newcommand{\cP}{\ensuremath{{c_{\scriptstyle P}}}}
\newcommand{\cV}{\ensuremath{{c_{\scriptstyle V}}}}
\newcommand{\lmix}{\ensuremath{{\ell_\mathrm{m}}}}
\newcommand{\HP}{\ensuremath{{H_{\scriptstyle P}}}}
\newcommand{\vel}{\ensuremath{{\upsilon}}}
\newcommand{\vconv}{\ensuremath{{\vel_\mathrm{c}}}}
\newcommand{\vsound}{\ensuremath{{\vel_\mathrm{s}}}}
\newcommand{\veq}{\ensuremath{{\vel_\mathrm{eq}}}}
\newcommand{\vect}[1]{\mbox{\boldmath $#1$}}
\newcommand{\gradient}{\mbox{\boldmath $\nabla$}}

\newcommand{\Chi}{\ensuremath{\raisebox{0.4ex}{$\chi$}}}

\newcommand{\kmsMpc}{\ensuremath{%
  \mathrm{km}\,\mathrm{s}^{-1}\,\mathrm{Mpc}^{-1}}}

\newcommand{\arrow}{\ensuremath{\Rightarrow}}
\newcommand{\lsim}{\mathrel{\hbox{\rlap{\lower.55ex \hbox {$\sim$}}
 \kern-.3em \raise.4ex \hbox{$<$}}}}
\newcommand{\gsim}{\mathrel{\hbox{\rlap{\lower.55ex \hbox {$\sim$}}
 \kern-.3em \raise.4ex \hbox{$>$}}}}

\newcommand{\beq}{\begin{equation}}
\newcommand{\eeq}{\end{equation}}
\newcommand{\beqar}{\begin{eqnarray}}
\newcommand{\eeqar}{\end{eqnarray}}
\newcommand{\der}{\ensuremath{\mathrm{d}}}
\newcommand{\dd}{\partial}

\newcommand{\half}{\ensuremath{{\textstyle\frac{1}{2}}}}
\newcommand{\third}{\ensuremath{{\textstyle\frac{1}{3}}}}
\newcommand{\onesixth}{\ensuremath{{\textstyle\frac{1}{6}}}}

\newcommand{\amu}{\ensuremath{m_\mathrm{u}}}
\newcommand{\e}{\ensuremath{\mathrm{e}}}
\newcommand{\n}{\ensuremath{\mathrm{n}}}
\newcommand{\p}{\ensuremath{\mathrm{p}}}
\renewcommand{\H}[1]{\ensuremath{{^{#1}\mathrm{H}}}}
\newcommand{\D}{\ensuremath{\mathrm{D}}}
\newcommand{\He}[1]{\ensuremath{{^{#1}\mathrm{He}}}}
\newcommand{\Li}[1]{\ensuremath{{^{#1}\mathrm{Li}}}}
\newcommand{\Be}[1]{\ensuremath{{^{#1}\mathrm{Be}}}}
\newcommand{\B}[1]{\ensuremath{{^{#1}\mathrm{B}}}}
\newcommand{\C}[1]{\ensuremath{{^{#1}\mathrm{C}}}}
\newcommand{\N}[1]{\ensuremath{{^{#1}\mathrm{N}}}}
\renewcommand{\O}[1]{\ensuremath{{^{#1}\mathrm{O}}}}
\newcommand{\F}[1]{\ensuremath{{^{#1}\mathrm{F}}}}
\newcommand{\Na}[1]{\ensuremath{{^{#1}\mathrm{Na}}}}
\newcommand{\Ne}[1]{\ensuremath{{^{#1}\mathrm{Ne}}}}
\newcommand{\Mg}[1]{\ensuremath{{^{#1}\mathrm{Mg}}}}
\newcommand{\Al}[1]{\ensuremath{{^{#1}\mathrm{\!Al}}}}
\newcommand{\Si}[1]{\ensuremath{{^{#1}\mathrm{Si}}}}
\newcommand{\Fe}[1]{\ensuremath{{^{#1}\mathrm{Fe}}}}
\newcommand{\el}[2]{\ensuremath{{^{#2}\mathrm{#1}}}}
\newcommand{\avg}[1]{\ensuremath{\langle#1\rangle}}
\newcommand{\rate}[1]{\ensuremath{r_\mathrm{#1}}}
\newcommand{\sv}[1]{\ensuremath{\langle\sigma v\rangle_{#1}}}
\newcommand{\svpp}{\ensuremath{\langle\sigma v\rangle_\mathrm{pp}}}
\newcommand{\svpD}{\ensuremath{\langle\sigma v\rangle_\mathrm{pD}}}
\newcommand{\life}[2]{\ensuremath{\tau_\mathrm{#1}(\mathrm{#2})}}
\newcommand{\fpp}[1]{\ensuremath{f_\mathrm{pp#1}}}


\begin{document}

\hints{2}

\begin{enumerate}

\item {\bf Density profile}
  \begin{enumerate}
    \item $m(r) = 4\pi\rho_cr^3[\frac{1}{3} - \frac{1}{5}(r/R)^2]$
    \item $M = \frac{8}{15} \pi \rho_c R^3$ 
    \item Hint: density = mass/volume
 \end{enumerate}

  
\item {\bf Hydrostatic equilibrium}

 \begin{enumerate}
    
 \item Gravity, pressure gradient.

 \item See Section~2.2.
   
 \item Hint: take limit for $dr \rightarrow 0$.
   
 \item Hint: use $P(R) = 0$ and the fact that $r \leq R$.
   
 \item Hint: calculate $m(r)$, substitute it directly in the equation
   for hydrostatic equilibrium and integrate from the surface to the
   center to obtain $P_c$. Answer: $P_c = \frac{15}{16\pi} GM^2/R^4$.
   
 \end{enumerate}


\item {\bf The virial theorem}  

  \begin{enumerate}

  \item Use eq.~(2.20) and the appropriate expression for $m(r)$.
    You should get: \\[1ex]
    for constant density: ~ 
    $E_{\rm gr} = \displaystyle -\frac{3}{5} \frac{G M^2}{R}$ \\[1ex]
    for $\rho(r)$ as in Ex.~2.1: ~
    $E_{\rm gr} = \displaystyle -\frac{5}{7} \frac{G M^2}{R}$

%    \begin {eqnarray*}
%      E_{\rm gr} &=& -\int_0^{M_*} \frac{G m(r)}{r}dm  \\
%      &=& -\int_0^{R_*} \frac{G \frac{4}{3}\pi r^3 \rho}{r} 4 \pi r^2 
%      \rho dr \,\,\rm{assuming}\,\,\rm{constant} \,\, \rho  \\
%      &=& -\frac{3}{5} \frac{G M_*^2}{R_*} \approx  -2.3.10^{48} \rm{erg}    \\
%    \end{eqnarray*}
    
  \item Hint: The internal kinetic energy per particle for an ideal
    gas is $\frac{3}{2}kT$ and the number of particles per unit volume
    can be written as $\rho/(\mu m_u)$, where $\mu$ is the mean
    molecular weight expressed in atomic units and $m_u$ the mass of
    one atomic unit.
    
    Now express the kinetic internal energy per volume. Integrate this
    over the full star.
    
  
  \item $E_{\mathrm{int},\odot} = \frac{3}{2} (k / m_u \mu) M \langle T
    \rangle \approx 2.1\times10^{48}$ erg, assuming $\mu = 0.6$. 

    $E_{\mathrm{gr},\odot} \approx -2.3 \times 10^{48}$~erg.

    $E_{\mathrm{tot},\odot} \approx -2\times 10^{47}$~erg, and the Sun would be
    (barely) bound. (Note that these estimates are very rough and can
    only be trusted to an order of magnitude!)
    
  \item -
    
  \item For the left-hand side of equation (2.40) substitute $P$ using the
  ideal gas law: $P = \rho k T / (\mu m_u) $, now compare with
  equation (2.39). 
  
  \item Use equation (2.40).
    
  \item Use (c) to estimate $\Delta E_{\rm int}$ and the virial theorem
  to estimate $\Delta E_{\rm tot}$.

 

 %   \item   $E_{\rm gr} = -E_{\rm int}$
      
 %   \item  $E_{\rm tot} = 0$, the star is not bound.

  \end{enumerate}
  
\item {\bf Conceptual questions}
  \begin{enumerate}
  \item {\it Use the virial theorem to explain why stars are hot,
    i.e. have a high internal temperature and therefore radiate energy.}
    
    For the gravitational energy $E_{\rm gr}$ of a star we know that 
    \begin{equation}
      E_{\rm gr} = -\alpha\frac{G M^2}{R} < 0,
    \end{equation}
    were $\alpha>0$ depends on the density distribution $\rho(r)$. Using
    the virial theorem $E_{\rm int} = -0.5 E_{\rm gr} > 0$, therefore
    $T>0$. Stellar gas is hot, therefore it must radiate, i.e.\ lose
    energy form its surface.

  \item {\it What are the consequences of energy loss for the star,
      especially for its temperature?}
    
    Energy loss: $\displaystyle \frac{dE_{\rm tot}}{dt} = -L < 0$.
    Using the virial theorem:
    \begin{equation}
      \begin{array}{lccccl}
        \displaystyle
        \frac{dE_{\rm gr}}{dt} &=& -2L& <& 0& \,\,\rm{star \, \,contracts} \\[2ex]
        \displaystyle
        \frac{dE_{\rm int}}{dt}& =& L&  >& 0& \,\,\rm{star} \,\, \rm{heats} \,\, \rm {up} \\
      \end{array}
    \end{equation}
    The energy liberated from contraction is used for heating the star,
    the other half is radiated away.

  \item {\it Most stars are in thermal equilibrium. What is
      compensating for the energy loss?}
    
    Energy liberated in nuclear reactions, $L =
    \displaystyle\frac{dE_{\rm nuc}}{dt}$.
    
  \item {\it What happens to a star in thermal equilibrium (and in
      hydrostatic equilibrium ) if the energy production by nuclear
      reactions in a star drops (slowly enough to maintain hydrostatic
      equilibrium)?}
    
    If $|dE_{\rm nuc}/dt| < L$ then $E_{\rm tot} \downarrow $. Using
    the virial theorem (which we can use because the star is still in
    hydrostatic equilibrium), also $|E_{\rm gr}| \uparrow $ (the star
    contracts) and $E_{\rm int} \uparrow $ (the star heats up).

  \item {\it Why does this have a stabilizing effect? On what time
      scale does the change take place.}

    The temperature increase will accelerate the nuclear reactions
    until $L = |dE_{\rm nuc}/dt|$ again. The changes take place on the
    Kelvin Helmholtz timescale.

  \item {\it What happens if hydrostatic equilibrium is violated, e.g.
      by a sudden increase of the pressure.}
    
    The virial theorem does not hold any longer and $d^2r/dt^2$ in the
    equation of motion is not zero anymore.  An increase in pressure
    will lift the gas elements outwards, i.e.  the star expands.  In a
    normal star the expansion will lead to a shallower pressure
    gradient. At the same time the gravitional acceleration is
    reduced. If hydrostatical equilibrium can be restored or not
    depends on the equation of state (see coming lectures). In a
    normal main sequence star equilibrium is restored on the
    hydrodynamical timescale.

    In extreme cases the star will explode. (For example when nuclear
    burning starts in a degenerate star, leading to a thermal nuclear
    runaway which suddenly lifts the degeneracy, such that the
    pressure becomes temperature dependent, which leads to an enormous
    increase of pressure; see coming lectures).




  \end{enumerate}


\end{enumerate}
\end{document}

\item {\bf Three important timescales in stellar evolution}
  
  \begin{enumerate}
  \item  {\it Nuclear timescale $\tau_{\rm n}$} 
    \begin{enumerate}
    \item Derive an expression for the nuclear timescale in solar
      units, i.e. expressed in terms of $R/\Rsun$, $M/\Msun$ and
      $L/\Lsun$. Show your derivation and your assumptions. 
      
    \item Use the mass-radius and mass-luminosity relations for main
      sequence stars express the nuclear timescale of main-sequence
      stars as function of the mass of the star only.
      
    \item Describe in your own words the meaning of the nuclear
      timescale.       

    \end{enumerate}

  \item {\it Thermal timescale $\tau_{\rm KH}$} 
    \begin{enumerate}
      
      In the time of Kelvin (1862) the energy source of the sun was
      unknown and the gravitational potential energy of the sun was
      proposed to be the main energy source. William Thomson (better
      known as Lord Kelvin) estimated the age of the sun and imposed
      an upper limit to the age of the earth which was in conflict
      with the new theory put forward by Charles Darwin (in
      1859). This theory required that the earth had to be much older,
      to account for the slow evolution of species. After a long and
      intense debate between the two scientist Darwin remained
      convinced that, in time, physicists would change their minds.

    \item Answer question a) i, ii and iii for the thermal timescale
      and calculate the age of the sun according to Kelvin.
      \addtocounter{enumiii}{2}
      
    \item Why are most stars observed to be main-sequence stars and
      why is the Hertzsprung-gap called a gap?

    \end{enumerate}

  \item  {\it Dynamical timescale $\tau_{\rm d}$} 
    \begin{enumerate}

    \item Answer question a) i, ii and iii for the dynamical timescale.
      \addtocounter{enumiii}{2}

    \item In stellar evolution models one often assumes that stars
      evolve {\it quasi-statically}, i.e. that the star is in
      hydrostatic equilibrium throughout. Why can we make that assumption?

    \item Rapid changes that are some ties observed in stars may
      indicate that dynamical processes are taking place. From the
      timescales of such changes - usually oscillations with a
      characteristic period - we may roughly estimate the average
      density of the Star. The sun has been observed to oscillate with
      a period of minutes, white dwarfs with periods of a few tens of
      seconds. Estimate the average density for the sun and for white
      dwarfs.
    \end{enumerate}

  \item {\it Comparison}
    \begin{enumerate}
    \item Summarize your results of the questions above by completing
      Table~\ref{tab:timescales} the timescale for a $1, 10, 25 \Msun$
      star in Table~\ref{tab:timescales}.
	
      \begin{table}[h]
	\center  
	\begin{tabular*}{0.75\textwidth}%
	  {@{\extracolsep{\fill}} ||l|l|l|l||}

	  \hline
	  \hline
	  & $\tau_{\rm n}$ & $\tau_{\rm_{\rm KH}}$ & $\tau_{\rm d}$
	  \\ 
	  \hline
	  & & &  \\
	  $1 \Msun$ & & &  \\
	   & & &  \\
	  \hline
	  & & &  \\
	  $10 \Msun$ & & &  \\
	  & & &  \\
	  \hline
	  & & &  \\
	  $25 \Msun$ & & &  \\
	  & & &  \\
	  \hline 
	  & & &  \\
	  $\rm{RG}~~ 1 \Msun$ & & &  \\
	  & & &  \\
	  \hline
	  \hline
	  
	\end {tabular*}
	\caption{\label{tab:timescales}}
      \end{table}

      
    \item For each of the following evolutionary stages indicate on
      which timescale they occur: {\it pre-main sequence contraction,
      supernova explosion, core hydrogen burning, core helium burning}.

      
    \item When the Sun becomes a red giant (RG), its radius will
      increase to $200\Rsun$ and its luminosity to
      $3000\Lsun$. Estimate $\tau_{hyd}$ and $\tau_{KH}$ for such a
      RG.

    \item How large would such a RG have to become for
      $\tau_{hyd}>\tau_{KH}$?  Assume both R and L increase at constant
      effective temperature ($T_{eff}$).
      
    \end{enumerate}
  \end{enumerate}
  



%\item {\bf (extra) Virial Theorem for Stars dominated by radiation pressure}
%  \begin{enumerate}

%  \end{enumerate}




