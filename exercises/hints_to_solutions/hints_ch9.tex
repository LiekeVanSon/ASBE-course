\documentclass[11pt,a4paper,fleqn]{report}

\usepackage{epsfig}
\usepackage{amsmath}
\usepackage{amssymb}
\usepackage{txfonts}

\textwidth 15.5cm
\textheight 23.0cm
\oddsidemargin 0cm
\evensidemargin 0cm
\topmargin 0cm

\pagestyle{empty}

\newcommand{\hints}[1]{\centerline{{\large\bf 
      Stellar Evolution -- Hints to exercises -- Chapter #1}} \bigskip
\renewcommand{\labelenumi}{#1.\arabic{enumi}}}

% for making counters of style 1.1, 1.2, etc
%\renewcommand{\labelenumi}{1.\arabic{enumi}}

% symbol definitions:
\newcommand{\Msun}{\ensuremath{{M}_\odot}}
\newcommand{\Lsun}{\ensuremath{{L}_\odot}}
\newcommand{\Rsun}{\ensuremath{{R}_\odot}}
\newcommand{\Mcore}{\ensuremath{M_\mathrm{c}}}
\newcommand{\MCh}{\ensuremath{M_\mathrm{Ch}}}
\newcommand{\Teff}{\ensuremath{T_\mathrm{eff}}}
\newcommand{\lum}{\ensuremath{l}}
\newcommand{\ledd}{\ensuremath{l_\mathrm{Edd}}}
\newcommand{\Ledd}{\ensuremath{L_\mathrm{Edd}}}
\newcommand{\Prad}{\ensuremath{P_\mathrm{rad}}}
\newcommand{\Pgas}{\ensuremath{P_\mathrm{gas}}}
\newcommand{\Pion}{\ensuremath{P_\mathrm{ion}}}
\newcommand{\Prot}{\ensuremath{P_\mathrm{rot}}}
\newcommand{\GammaC}{\ensuremath{{\Gamma_\mathrm{C}}}}
\newcommand{\gammad}{\ensuremath{{\gamma_\mathrm{ad}}}}
\newcommand{\gradad}{\ensuremath{{\nabla_\mathrm{ad}}}}
\newcommand{\gradrad}{\ensuremath{{\nabla_\mathrm{rad}}}}
\newcommand{\tdyn}{\ensuremath{{\tau_\mathrm{dyn}}}}
\newcommand{\tKH}{\ensuremath{{\tau_\mathrm{KH}}}}
\newcommand{\tnuc}{\ensuremath{{\tau_\mathrm{nuc}}}}
\newcommand{\tms}{\ensuremath{{\tau_\mathrm{MS}}}}
\newcommand{\enuc}{\ensuremath{{\epsilon_\mathrm{nuc}}}}
\newcommand{\eneu}{\ensuremath{{\epsilon_\nu}}}
\newcommand{\egrav}{\ensuremath{{\epsilon_\mathrm{gr}}}}
\newcommand{\cP}{\ensuremath{{c_{\scriptstyle P}}}}
\newcommand{\cV}{\ensuremath{{c_{\scriptstyle V}}}}
\newcommand{\lmix}{\ensuremath{{\ell_\mathrm{m}}}}
\newcommand{\HP}{\ensuremath{{H_{\scriptstyle P}}}}
\newcommand{\vel}{\ensuremath{{\upsilon}}}
\newcommand{\vconv}{\ensuremath{{\vel_\mathrm{c}}}}
\newcommand{\vsound}{\ensuremath{{\vel_\mathrm{s}}}}
\newcommand{\veq}{\ensuremath{{\vel_\mathrm{eq}}}}
\newcommand{\vect}[1]{\mbox{\boldmath $#1$}}
\newcommand{\gradient}{\mbox{\boldmath $\nabla$}}

\newcommand{\Chi}{\ensuremath{\raisebox{0.4ex}{$\chi$}}}

\newcommand{\kmsMpc}{\ensuremath{%
  \mathrm{km}\,\mathrm{s}^{-1}\,\mathrm{Mpc}^{-1}}}

\newcommand{\arrow}{\ensuremath{\Rightarrow}}
\newcommand{\lsim}{\mathrel{\hbox{\rlap{\lower.55ex \hbox {$\sim$}}
 \kern-.3em \raise.4ex \hbox{$<$}}}}
\newcommand{\gsim}{\mathrel{\hbox{\rlap{\lower.55ex \hbox {$\sim$}}
 \kern-.3em \raise.4ex \hbox{$>$}}}}

\newcommand{\beq}{\begin{equation}}
\newcommand{\eeq}{\end{equation}}
\newcommand{\beqar}{\begin{eqnarray}}
\newcommand{\eeqar}{\end{eqnarray}}
\newcommand{\der}{\ensuremath{\mathrm{d}}}
\newcommand{\dd}{\partial}

\newcommand{\half}{\ensuremath{{\textstyle\frac{1}{2}}}}
\newcommand{\third}{\ensuremath{{\textstyle\frac{1}{3}}}}
\newcommand{\onesixth}{\ensuremath{{\textstyle\frac{1}{6}}}}

\newcommand{\amu}{\ensuremath{m_\mathrm{u}}}
\newcommand{\e}{\ensuremath{\mathrm{e}}}
\newcommand{\n}{\ensuremath{\mathrm{n}}}
\newcommand{\p}{\ensuremath{\mathrm{p}}}
\renewcommand{\H}[1]{\ensuremath{{^{#1}\mathrm{H}}}}
\newcommand{\D}{\ensuremath{\mathrm{D}}}
\newcommand{\He}[1]{\ensuremath{{^{#1}\mathrm{He}}}}
\newcommand{\Li}[1]{\ensuremath{{^{#1}\mathrm{Li}}}}
\newcommand{\Be}[1]{\ensuremath{{^{#1}\mathrm{Be}}}}
\newcommand{\B}[1]{\ensuremath{{^{#1}\mathrm{B}}}}
\newcommand{\C}[1]{\ensuremath{{^{#1}\mathrm{C}}}}
\newcommand{\N}[1]{\ensuremath{{^{#1}\mathrm{N}}}}
\renewcommand{\O}[1]{\ensuremath{{^{#1}\mathrm{O}}}}
\newcommand{\F}[1]{\ensuremath{{^{#1}\mathrm{F}}}}
\newcommand{\Na}[1]{\ensuremath{{^{#1}\mathrm{Na}}}}
\newcommand{\Ne}[1]{\ensuremath{{^{#1}\mathrm{Ne}}}}
\newcommand{\Mg}[1]{\ensuremath{{^{#1}\mathrm{Mg}}}}
\newcommand{\Al}[1]{\ensuremath{{^{#1}\mathrm{\!Al}}}}
\newcommand{\Si}[1]{\ensuremath{{^{#1}\mathrm{Si}}}}
\newcommand{\Fe}[1]{\ensuremath{{^{#1}\mathrm{Fe}}}}
\newcommand{\el}[2]{\ensuremath{{^{#2}\mathrm{#1}}}}
\newcommand{\avg}[1]{\ensuremath{\langle#1\rangle}}
\newcommand{\rate}[1]{\ensuremath{r_\mathrm{#1}}}
\newcommand{\sv}[1]{\ensuremath{\langle\sigma v\rangle_{#1}}}
\newcommand{\svpp}{\ensuremath{\langle\sigma v\rangle_\mathrm{pp}}}
\newcommand{\svpD}{\ensuremath{\langle\sigma v\rangle_\mathrm{pD}}}
\newcommand{\life}[2]{\ensuremath{\tau_\mathrm{#1}(\mathrm{#2})}}
\newcommand{\fpp}[1]{\ensuremath{f_\mathrm{pp#1}}}


\begin{document}

\hints{9}

\begin{enumerate}
  
\item {\bf Kippenhahn diagram of ZAMS}

  Discuss your answers to this question with your fellow students or
  with the teaching assistant.

  \begin{enumerate}

  \item Hint: Low-mass stars are dominated by bound-free and free-free
    absorption, for which $\kappa$ is higher the lower the
    temperature.

  \item Hint: for $M > 1.3 \Msun$ the CNO-cycle dominates (see Section
    9.2.2).

  \item See Section 9.2.2.

  \end{enumerate}


\item {\bf Conceptual questions}

  \begin{enumerate}
	
  \item See Section 9.1.1.  Convection is so efficient that a fully
    convective star is capable of transporting any luminosity,
    independent of its structure -- unlike a radiative star where $L$
    is strongly coupled to the $T$-gradient.
    
  \item See Section 9.1.1.

  \item See Sections 7.4.1 and 9.2.

  \item See Section 9.2.

  \end{enumerate} 


\item { \bf Central Temperature vs Mass}

  First derive the homology relation for the radius, eq.~(7.36),
  assuming that the stars are homogeneous and have constant opacity,
  i.e.\ by using (7.32) for the luminosity and equating this to the
  luminosity produced by nuclear reactions. Then substitute the radius
  into the homology relation (7.28) for the central temperature to
  find (7.38), i.e.\ $T_c \propto \mu^{7/(\nu +3)} M^{4/(\nu +3)}$.
  Take $\nu = 4$ for pp burning ($M \leq 1.3 \Msun$) and $\nu = 18$
  for CNO burning ($M \geq 1.3 \Msun$).

  To find the proportionality constant use solar values for pp burning
  and demand that $T_c$ is a continous function to match the pp and
  CNO parts of the relation. In detail: for $M \leq 1.3 \Msun$, $T_c =
  T_{c,\odot}\, (M/\Msun)^{4/7}$ and for $M \geq 1.3 \Msun$, $T_c = C\,
  (M/\Msun)^{4/21}$. Find a numerical expression for $C$ by demanding
  that $T_{\rm pp} = T_{\rm CNO}$ for $M = 1.3\Msun$, which gives $C =
  1.43\times10^7$\,K.
  \[
  T = \begin{cases}
    1.30 \times 10^7\, (M/\Msun)^{4/7}~\mathrm{K} & \text{if $M \leq 1.3 
      \Msun$},\\
    1.43 \times 10^7\, (M/\Msun)^{4/21}~\mathrm{K} & \text{if $M > 1.3
      \Msun$}.
  \end{cases}
  \]


\item {\bf Mass-luminosity relation}

  Assuming energy transport by radiation gives $T \propto M \kappa l
  /(T^3 R^4)$ (eq.~7.29). Substitute $\kappa \propto \rho T^{-7/2}$;
  eliminate $\rho$ by $\rho \propto M/R^3$; and eliminate $T$ using $T
  \propto \mu M/R$ (eq.~7.28).  This altogether will give you
  eq.~(7.33), i.e.\
  \begin{equation}
    L \propto \mu^{7.5} M^{5.5} R^{-0.5}. \label{eq:l1}
  \end{equation}
  Now obtain a homology relation for $R$ using $\der l/\der m =
  \epsilon \propto \rho T^4$, i.e.\ $L \propto M \rho T^4$, again
  eliminating $T$ and $\rho$. This will give you $R \propto \mu^{0}
  M^{3/7}$ (compare to eq.~7.36). Use this relation to express
  (\ref{eq:l1}) in terms of $M$ and $\mu$ only:
  \[ L \propto \mu^{7.5} M^{5.3}. \]


\end{enumerate} 

\end{document}


%%%%%%%%%%%%%%%


  \item {\bf Homology relation: more formal derivation}
    Now we will derive one of the homology
    relations from the hydrostatic equilibrium equation step by step:
    \begin{equation}
      \frac{dP}{dm} = -\frac{G}{4\pi}\frac{m}{r^4} \label{he}
    \end{equation}
    Let $\xi$ be the relative mass coordinate, $\xi = \frac{m}{M}$.
    
    \begin{enumerate}
    \item Show that you can rewrite equation~\ref{he} as:
      \[
      \frac{dP}{d\xi} = -\frac{G}{4\pi}\frac{\xi M^2}{r^4}
      \]
      Now define
     % \[
     $ \frac{M}{M'} = x, \,\, \frac{r}{r'} = z, \,\, \frac{P}{P'} = p.$
     % \]
    \item Show that you can rewrite equation~\ref{he} as:
      \[
      \frac{dP'}{d\xi} = -\frac{G}{4\pi}\frac{\xi M'^2}{r'^4} \biggl[  \frac{x^2}{z^4 p} \biggl]
      \]
    \item Explain why $\frac{x^2}{z^4 p} = 1$. 
    \item For homologous stars
      $\frac{r}{r'} = \frac{R}{R'}$. Use iii) to express $P/P'$ as function of
      $M/M'$ and $R/R'$.
  
  \item Use the same method to derive an expression for $\rho/\rho'$ from the
    equation of mass conservation.
    
  \item (At home) Read page 191 of Kippenhahn and Weigert. Convince
    yourself that the homology condition, equation 20.2, for a Star $S$
    and a Star $S'$ comes down to saying that Star $S$ and Star $S'$ have
    a similar density distribution.

    \end{enumerate}  


\item Show that the above relation is valid for all polytropes
      $P = K \rho^\gamma =  K \rho^{(n+1)/n}$.

      \item Give an expression for $C$ as function of $n$.

      \item In the practicum 2 you saw that polytropic models with
      index n in the range 1.5 ... 3 are a reasonable approximation
      for most (detailed models of) actual stars. Between what
      numerical values will $C$ vary for actual stars.
