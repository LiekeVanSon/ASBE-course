\documentclass[11pt,a4paper,fleqn]{report}

\usepackage{epsfig}
\usepackage{amsmath}
\usepackage{amssymb}
\usepackage{txfonts}

\textwidth 15.5cm
\textheight 23.0cm
\oddsidemargin 0cm
\evensidemargin 0cm
\topmargin 0cm

\pagestyle{empty}

\newcommand{\hints}[1]{\centerline{{\large\bf 
      Stellar Evolution -- Hints to exercises -- Chapter #1}} \bigskip
\renewcommand{\labelenumi}{#1.\arabic{enumi}}}

% for making counters of style 1.1, 1.2, etc
%\renewcommand{\labelenumi}{1.\arabic{enumi}}

% symbol definitions:
\newcommand{\Msun}{\ensuremath{{M}_\odot}}
\newcommand{\Lsun}{\ensuremath{{L}_\odot}}
\newcommand{\Rsun}{\ensuremath{{R}_\odot}}
\newcommand{\Mcore}{\ensuremath{M_\mathrm{c}}}
\newcommand{\MCh}{\ensuremath{M_\mathrm{Ch}}}
\newcommand{\Teff}{\ensuremath{T_\mathrm{eff}}}
\newcommand{\lum}{\ensuremath{l}}
\newcommand{\ledd}{\ensuremath{l_\mathrm{Edd}}}
\newcommand{\Ledd}{\ensuremath{L_\mathrm{Edd}}}
\newcommand{\Prad}{\ensuremath{P_\mathrm{rad}}}
\newcommand{\Pgas}{\ensuremath{P_\mathrm{gas}}}
\newcommand{\Pion}{\ensuremath{P_\mathrm{ion}}}
\newcommand{\Prot}{\ensuremath{P_\mathrm{rot}}}
\newcommand{\GammaC}{\ensuremath{{\Gamma_\mathrm{C}}}}
\newcommand{\gammad}{\ensuremath{{\gamma_\mathrm{ad}}}}
\newcommand{\gradad}{\ensuremath{{\nabla_\mathrm{ad}}}}
\newcommand{\gradrad}{\ensuremath{{\nabla_\mathrm{rad}}}}
\newcommand{\tdyn}{\ensuremath{{\tau_\mathrm{dyn}}}}
\newcommand{\tKH}{\ensuremath{{\tau_\mathrm{KH}}}}
\newcommand{\tnuc}{\ensuremath{{\tau_\mathrm{nuc}}}}
\newcommand{\tms}{\ensuremath{{\tau_\mathrm{MS}}}}
\newcommand{\enuc}{\ensuremath{{\epsilon_\mathrm{nuc}}}}
\newcommand{\eneu}{\ensuremath{{\epsilon_\nu}}}
\newcommand{\egrav}{\ensuremath{{\epsilon_\mathrm{gr}}}}
\newcommand{\cP}{\ensuremath{{c_{\scriptstyle P}}}}
\newcommand{\cV}{\ensuremath{{c_{\scriptstyle V}}}}
\newcommand{\lmix}{\ensuremath{{\ell_\mathrm{m}}}}
\newcommand{\HP}{\ensuremath{{H_{\scriptstyle P}}}}
\newcommand{\vel}{\ensuremath{{\upsilon}}}
\newcommand{\vconv}{\ensuremath{{\vel_\mathrm{c}}}}
\newcommand{\vsound}{\ensuremath{{\vel_\mathrm{s}}}}
\newcommand{\veq}{\ensuremath{{\vel_\mathrm{eq}}}}
\newcommand{\vect}[1]{\mbox{\boldmath $#1$}}
\newcommand{\gradient}{\mbox{\boldmath $\nabla$}}

\newcommand{\Chi}{\ensuremath{\raisebox{0.4ex}{$\chi$}}}

\newcommand{\kmsMpc}{\ensuremath{%
  \mathrm{km}\,\mathrm{s}^{-1}\,\mathrm{Mpc}^{-1}}}

\newcommand{\arrow}{\ensuremath{\Rightarrow}}
\newcommand{\lsim}{\mathrel{\hbox{\rlap{\lower.55ex \hbox {$\sim$}}
 \kern-.3em \raise.4ex \hbox{$<$}}}}
\newcommand{\gsim}{\mathrel{\hbox{\rlap{\lower.55ex \hbox {$\sim$}}
 \kern-.3em \raise.4ex \hbox{$>$}}}}

\newcommand{\beq}{\begin{equation}}
\newcommand{\eeq}{\end{equation}}
\newcommand{\beqar}{\begin{eqnarray}}
\newcommand{\eeqar}{\end{eqnarray}}
\newcommand{\der}{\ensuremath{\mathrm{d}}}
\newcommand{\dd}{\partial}

\newcommand{\half}{\ensuremath{{\textstyle\frac{1}{2}}}}
\newcommand{\third}{\ensuremath{{\textstyle\frac{1}{3}}}}
\newcommand{\onesixth}{\ensuremath{{\textstyle\frac{1}{6}}}}

\newcommand{\amu}{\ensuremath{m_\mathrm{u}}}
\newcommand{\e}{\ensuremath{\mathrm{e}}}
\newcommand{\n}{\ensuremath{\mathrm{n}}}
\newcommand{\p}{\ensuremath{\mathrm{p}}}
\renewcommand{\H}[1]{\ensuremath{{^{#1}\mathrm{H}}}}
\newcommand{\D}{\ensuremath{\mathrm{D}}}
\newcommand{\He}[1]{\ensuremath{{^{#1}\mathrm{He}}}}
\newcommand{\Li}[1]{\ensuremath{{^{#1}\mathrm{Li}}}}
\newcommand{\Be}[1]{\ensuremath{{^{#1}\mathrm{Be}}}}
\newcommand{\B}[1]{\ensuremath{{^{#1}\mathrm{B}}}}
\newcommand{\C}[1]{\ensuremath{{^{#1}\mathrm{C}}}}
\newcommand{\N}[1]{\ensuremath{{^{#1}\mathrm{N}}}}
\renewcommand{\O}[1]{\ensuremath{{^{#1}\mathrm{O}}}}
\newcommand{\F}[1]{\ensuremath{{^{#1}\mathrm{F}}}}
\newcommand{\Na}[1]{\ensuremath{{^{#1}\mathrm{Na}}}}
\newcommand{\Ne}[1]{\ensuremath{{^{#1}\mathrm{Ne}}}}
\newcommand{\Mg}[1]{\ensuremath{{^{#1}\mathrm{Mg}}}}
\newcommand{\Al}[1]{\ensuremath{{^{#1}\mathrm{\!Al}}}}
\newcommand{\Si}[1]{\ensuremath{{^{#1}\mathrm{Si}}}}
\newcommand{\Fe}[1]{\ensuremath{{^{#1}\mathrm{Fe}}}}
\newcommand{\el}[2]{\ensuremath{{^{#2}\mathrm{#1}}}}
\newcommand{\avg}[1]{\ensuremath{\langle#1\rangle}}
\newcommand{\rate}[1]{\ensuremath{r_\mathrm{#1}}}
\newcommand{\sv}[1]{\ensuremath{\langle\sigma v\rangle_{#1}}}
\newcommand{\svpp}{\ensuremath{\langle\sigma v\rangle_\mathrm{pp}}}
\newcommand{\svpD}{\ensuremath{\langle\sigma v\rangle_\mathrm{pD}}}
\newcommand{\life}[2]{\ensuremath{\tau_\mathrm{#1}(\mathrm{#2})}}
\newcommand{\fpp}[1]{\ensuremath{f_\mathrm{pp#1}}}


\begin{document}

\hints{7}

\begin{enumerate}


\item {\bf General understanding of the stellar evolution equations} 

  Most answers can be found in the lecture notes, Section 7.1 and 7.3.



\item {\bf Dynamical Stability} 

  \begin{enumerate}

  \item See Section 7.4.
    
  \item See Section 7.5.1.
    
  \item Stars dominated by radiation pressure (high $T$ / low $\rho$)
    or stars where the electrons are extremely relativistic and
    degenerate. Small disturbances can lead to either collapse or
    explosion of the star.
    
  \item Partial ionization can lead to $\gamma_{\rm ad} < 4/3$.
   
  \item Fe photo-disintegration can occur in all massive stars that
    form an Fe core after going through successive nuclear burning
    cycles, and initiates the collapse of the core and a supernova
    explosion. Pair creation can occur in extremely massive stars.

  \end{enumerate}


\item {\bf Mass radius relation for degenerate stars}
  \begin{enumerate}

  \item $\displaystyle P_c = b\, \frac{G M^2}{R^4}$ \quad and also
    \quad $\displaystyle P_c = K_\mathrm{NR}\,
    \Bigg(\frac{\rho_c}{\mu_e}\Bigg)^{5/3} = K_\mathrm{NR}\,
    \Bigg(\frac{a\,\bar{\rho}}{\mu_e}\Bigg)^{5/3} = K_\mathrm{NR}\,
    \Bigg(\frac{a}{\mu_e}\,\frac{3 M}{4\pi R^3}\Bigg)^{5/3} $.

    Equating the two expressions gives \quad $\displaystyle R =
    \frac{K_\mathrm{NR}}{b G}\, \Bigg( \frac{3a}{4\pi\mu_e}
    \Bigg)^{5/3}\, M^{-1/3}$. 

    Therefore $R \propto M^{-1/3}$: the more massive the white dwarf,
    the smaller its radius.  (Note that $a$ and $b$ in this case
    correspond to the values for a $n=1.5$ polytrope, and can be found
    in Table~4.1.)

  \item A silimar derivation as above shows that the radius drops out,
    and \quad $\displaystyle M = \Bigg( \frac{K_\mathrm{NR}}{b G}
    \Bigg)^{3/2} \Bigg( \frac{3a}{4\pi\mu_e} \Bigg)^2$ is constant.
    Taking $a$ and $b$ for an $N=3$ polytrope (Table~4.1) gives the
    Cahndrasekhar mass, $M_{\rm Ch} = 1.46\, (2/\mu_e)^2\, \Msun$.

  \item SNe Ia are thought to occur in binaries where a white dwarf
    with a mass lower than $M_{\rm Ch}$ accretes from a companion
    star. As its mass increases and it radius decreases, the white
    dwarf becomes more dense, until all the electrons become extremely
    relativistic degenarate. The electron pressure cannot support more
    weight than $M_{\rm Ch}$ and the star collapses, ....

  \end{enumerate}


\item {\bf Main-sequence homology relations}

%  Discuss your answers to this question with your fellow students or
%  with the teaching assistant.
  \begin {enumerate}	
    \addtocounter{enumii}{1}
	
  \item Low-mass stars with $M \lsim 0.25\,\Msun$. Stars with $M \gsim
    1\,\Msun$ also have a roughly similar density dirtribution.
	
  \item An ideal gas is assumed, which breaks down for the
    highest-mass stars where radiation pressure is important.
    Radiative energy transport is assumed, which is not valid for
    low-mass stars which have large convective envelopes. $\kappa$ is
    assumed to be constant which is in reality a function of density
    and temperature.
	
  \item Use $a=1$ and $b= -3.5$ in equation (6.30). Use equation
    (6.30) and (6.31) to calculate the relation.
	
  \item $R \propto M^{0.7} \Rightarrow L \propto \mu^{7.5} M^{5.15}$.
    Opacity that follows Kramers' law is mostly due to free-free
    absorption. This is the main source of opacity inside low-mass
    stars. In more massive stars electron scattering is the main
    opacity source.
	
  \end{enumerate}
  

\item {\bf Central behaviour of the stellar structure equations}
  \begin{enumerate}
   
  \item For a variable $A$ write $\der A/\der r = \der A/\der m \cdot
    \der m/\der r$ with $\der m/\der r = 4\pi r^2 \rho$, etc. Apply
    this to eqs.~(7.12)--(7.15).

  \item Write each quantity $A$ as a Taylor expansion around the centre
    \[
    A(r) = A(0) + \frac{\der A}{\der r}\Bigg|_{r=0} \,r
    + \frac{1}{2} \frac{\der^2 A}{\der r^2}\Bigg|_{r=0} \,r^2
    + \frac{1}{6} \frac{\der^3 A}{\der r^3}\Bigg|_{r=0} \,r^3 
    + \dots
    \]
    and keep only the lowest-order term with $r^n$.
    % Hint: take $\rho \approx \rho_c$ close to the center
    
    \begin {itemize} 

    \item For the mass $m(r)$, use $\der m/\der r = 4\pi r^2 \rho$.
      Verify that $\der m/\der r$ and $\der^2 m/\der r^2$ both vanish
      at $r=0$, and that $\der^3 m/\der r^3 = 8\pi \rho_c$. Since
      $m(0) = 0$ you get
      \[
      m(r) = \frac{4}{3}\pi \rho_c\, r^3 + \ldots
      \]
      which  is simply the mass of a small sphere with constant
      density $\rho_c$.

    \item For the luminosity $\lum(r)$, it is easier to use
      $\der\lum/\der m = \epsilon \equiv \epsilon_\mathrm{nuc} -
      \epsilon_\nu + \epsilon_\mathrm{gr}$, and use a Taylor expansion
      in $m$ rather than $r$. This gives $\lum(r) = \epsilon_c\, m(r)
      + \ldots$ because $\lum(0) = 0$.  Then combine with the
      expansion for $m(r)$:
      \[
      \lum(r) = \frac{4}{3}\pi \rho_c\,\epsilon_c\, r^3 + \ldots
      \]

    \item For the pressure $P(r)$, use $\der P/\der r = Gm\rho/r^2$.
      Verify that $\der P/\der r = 0$ and $\der^2 P/\der r^2 =
      -\frac{4}{3}\pi G \rho_c^2$ at $r=0$. At the centre $P(0) = P_c$
      so that
      \[
      P(r) = P_c - \frac{2}{3}\pi G \rho_c^2 \, r^2 + \ldots
      \]

    \item For the temperature $T(r)$ we have to distinguish between
      radiative and convective transport.  In case of \emph{radiative
        transport}, take eq.~(5.16) for $\der T/\der r$ which you can
      write as
      \[
      \frac{\der (T^4)}{\der r} = 
      -\frac{3}{4\pi a c}\,\frac{\kappa\rho\lum}{r^2}
      \]
      which vanishes in the centre, since $\lum \propto r^3$ and both
      $\kappa_c$ and $\rho_c$ are finite. The second derivative of
      $T^4$ in the centre is
      \[
      \frac{\der^2 (T^4)}{\der r^2} = -\frac{3\kappa_c\rho_c}{4\pi a c}\,
      \Bigg(\frac{1}{r^2}\frac{\der\lum}{\der r} 
      - \frac{2}{r^3}\,\lum \Bigg)_{r=0}
      = -\frac{\kappa_c\,\rho_c^2\,\epsilon_c}{ac}
      \]
      so that
      \[
      T^4(r) = T_c^4 - \frac{\kappa_c\, \rho_c^2\,
        \epsilon_c}{2ac}\,r^2 + \ldots
      \]
      
    \item In case of \emph{convective energy transport} write $\der
      T/\der r = -(T/P) \, \gradad\, \der P/\der r$ which can be
      written as
      \[
      \frac{\der\ln T}{\der r} = \frac{\gradad}{P}\,\frac{\der P}{\der r}
      \]
      which vanishes in the centre because $\der P/\der r = 0$. The
      second derivative of $\ln T$ in the centre equals $\der^2 \ln
      T/\der r^2 = (\gradad/P)_c\,\der^2 P/\der r^2 = (\gradad/P)_c\,
      \frac{4}{3}\pi G \rho_c^2$, so that
      \[
      \ln T(r) = \ln T_c - \frac{2\pi G\, \rho_c^2\,
        \nabla_\mathrm{ad,c}}{3 P_c}\,r^2 + \ldots
      \]

    \end{itemize}

  \end{enumerate}

\end{enumerate}

\end{document}

