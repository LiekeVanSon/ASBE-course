\documentclass[11pt,a4paper]{report}

\usepackage{epsfig}
\usepackage{amsmath}
\usepackage{amssymb}
\usepackage{txfonts}

\textwidth 15.5cm
\textheight 23.0cm
\oddsidemargin 0cm
\evensidemargin 0cm
\topmargin 0cm

\pagestyle{empty}

\newcommand{\hints}[1]{\centerline{{\large\bf 
      Stellar Evolution -- Hints to exercises -- Chapter #1}} \bigskip
\renewcommand{\labelenumi}{#1.\arabic{enumi}}}

% for making counters of style 1.1, 1.2, etc
%\renewcommand{\labelenumi}{1.\arabic{enumi}}

% symbol definitions:
\newcommand{\Msun}{\ensuremath{{M}_\odot}}
\newcommand{\Lsun}{\ensuremath{{L}_\odot}}
\newcommand{\Rsun}{\ensuremath{{R}_\odot}}
\newcommand{\Mcore}{\ensuremath{M_\mathrm{c}}}
\newcommand{\MCh}{\ensuremath{M_\mathrm{Ch}}}
\newcommand{\Teff}{\ensuremath{T_\mathrm{eff}}}
\newcommand{\lum}{\ensuremath{l}}
\newcommand{\ledd}{\ensuremath{l_\mathrm{Edd}}}
\newcommand{\Ledd}{\ensuremath{L_\mathrm{Edd}}}
\newcommand{\Prad}{\ensuremath{P_\mathrm{rad}}}
\newcommand{\Pgas}{\ensuremath{P_\mathrm{gas}}}
\newcommand{\Pion}{\ensuremath{P_\mathrm{ion}}}
\newcommand{\Prot}{\ensuremath{P_\mathrm{rot}}}
\newcommand{\GammaC}{\ensuremath{{\Gamma_\mathrm{C}}}}
\newcommand{\gammad}{\ensuremath{{\gamma_\mathrm{ad}}}}
\newcommand{\gradad}{\ensuremath{{\nabla_\mathrm{ad}}}}
\newcommand{\gradrad}{\ensuremath{{\nabla_\mathrm{rad}}}}
\newcommand{\tdyn}{\ensuremath{{\tau_\mathrm{dyn}}}}
\newcommand{\tKH}{\ensuremath{{\tau_\mathrm{KH}}}}
\newcommand{\tnuc}{\ensuremath{{\tau_\mathrm{nuc}}}}
\newcommand{\tms}{\ensuremath{{\tau_\mathrm{MS}}}}
\newcommand{\enuc}{\ensuremath{{\epsilon_\mathrm{nuc}}}}
\newcommand{\eneu}{\ensuremath{{\epsilon_\nu}}}
\newcommand{\egrav}{\ensuremath{{\epsilon_\mathrm{gr}}}}
\newcommand{\cP}{\ensuremath{{c_{\scriptstyle P}}}}
\newcommand{\cV}{\ensuremath{{c_{\scriptstyle V}}}}
\newcommand{\lmix}{\ensuremath{{\ell_\mathrm{m}}}}
\newcommand{\HP}{\ensuremath{{H_{\scriptstyle P}}}}
\newcommand{\vel}{\ensuremath{{\upsilon}}}
\newcommand{\vconv}{\ensuremath{{\vel_\mathrm{c}}}}
\newcommand{\vsound}{\ensuremath{{\vel_\mathrm{s}}}}
\newcommand{\veq}{\ensuremath{{\vel_\mathrm{eq}}}}
\newcommand{\vect}[1]{\mbox{\boldmath $#1$}}
\newcommand{\gradient}{\mbox{\boldmath $\nabla$}}

\newcommand{\Chi}{\ensuremath{\raisebox{0.4ex}{$\chi$}}}

\newcommand{\kmsMpc}{\ensuremath{%
  \mathrm{km}\,\mathrm{s}^{-1}\,\mathrm{Mpc}^{-1}}}

\newcommand{\arrow}{\ensuremath{\Rightarrow}}
\newcommand{\lsim}{\mathrel{\hbox{\rlap{\lower.55ex \hbox {$\sim$}}
 \kern-.3em \raise.4ex \hbox{$<$}}}}
\newcommand{\gsim}{\mathrel{\hbox{\rlap{\lower.55ex \hbox {$\sim$}}
 \kern-.3em \raise.4ex \hbox{$>$}}}}

\newcommand{\beq}{\begin{equation}}
\newcommand{\eeq}{\end{equation}}
\newcommand{\beqar}{\begin{eqnarray}}
\newcommand{\eeqar}{\end{eqnarray}}
\newcommand{\der}{\ensuremath{\mathrm{d}}}
\newcommand{\dd}{\partial}

\newcommand{\half}{\ensuremath{{\textstyle\frac{1}{2}}}}
\newcommand{\third}{\ensuremath{{\textstyle\frac{1}{3}}}}
\newcommand{\onesixth}{\ensuremath{{\textstyle\frac{1}{6}}}}

\newcommand{\amu}{\ensuremath{m_\mathrm{u}}}
\newcommand{\e}{\ensuremath{\mathrm{e}}}
\newcommand{\n}{\ensuremath{\mathrm{n}}}
\newcommand{\p}{\ensuremath{\mathrm{p}}}
\renewcommand{\H}[1]{\ensuremath{{^{#1}\mathrm{H}}}}
\newcommand{\D}{\ensuremath{\mathrm{D}}}
\newcommand{\He}[1]{\ensuremath{{^{#1}\mathrm{He}}}}
\newcommand{\Li}[1]{\ensuremath{{^{#1}\mathrm{Li}}}}
\newcommand{\Be}[1]{\ensuremath{{^{#1}\mathrm{Be}}}}
\newcommand{\B}[1]{\ensuremath{{^{#1}\mathrm{B}}}}
\newcommand{\C}[1]{\ensuremath{{^{#1}\mathrm{C}}}}
\newcommand{\N}[1]{\ensuremath{{^{#1}\mathrm{N}}}}
\renewcommand{\O}[1]{\ensuremath{{^{#1}\mathrm{O}}}}
\newcommand{\F}[1]{\ensuremath{{^{#1}\mathrm{F}}}}
\newcommand{\Na}[1]{\ensuremath{{^{#1}\mathrm{Na}}}}
\newcommand{\Ne}[1]{\ensuremath{{^{#1}\mathrm{Ne}}}}
\newcommand{\Mg}[1]{\ensuremath{{^{#1}\mathrm{Mg}}}}
\newcommand{\Al}[1]{\ensuremath{{^{#1}\mathrm{\!Al}}}}
\newcommand{\Si}[1]{\ensuremath{{^{#1}\mathrm{Si}}}}
\newcommand{\Fe}[1]{\ensuremath{{^{#1}\mathrm{Fe}}}}
\newcommand{\el}[2]{\ensuremath{{^{#2}\mathrm{#1}}}}
\newcommand{\avg}[1]{\ensuremath{\langle#1\rangle}}
\newcommand{\rate}[1]{\ensuremath{r_\mathrm{#1}}}
\newcommand{\sv}[1]{\ensuremath{\langle\sigma v\rangle_{#1}}}
\newcommand{\svpp}{\ensuremath{\langle\sigma v\rangle_\mathrm{pp}}}
\newcommand{\svpD}{\ensuremath{\langle\sigma v\rangle_\mathrm{pD}}}
\newcommand{\life}[2]{\ensuremath{\tau_\mathrm{#1}(\mathrm{#2})}}
\newcommand{\fpp}[1]{\ensuremath{f_\mathrm{pp#1}}}


\begin{document}

\hints{1}

\begin{enumerate}

%\item {\bf Evolutionary stages }
%-

\addtocounter{enumi}{1}  

\item {\bf Basic assumptions } 
  \begin{enumerate}
  \item $d \approx 6.10^7\Rsun$, $(R/d)^2 \approx 3\cdot10^{-16}$ times weaker. \\
    Hint: Think about mass transfer, collisions, mergers etcetera.
  \item Hint: which stars produce heavy elements and lose a lot of mass? 
  \item $a/g \approx 2\cdot10^{-5}$ at the equator.
  \end{enumerate}


\item {\bf Mass-luminosity and mass-radius relation} 
 
  \begin {enumerate}

  \item Hint: What temperature is needed to ignite H fusion?    

  \item See Section~1.1.
   
  \item $x \approx 3.8$, $y \approx 0.7$.
    
  \item Hint: Assume that a constant fraction of the total mass of the
    star can actually be used as fuel. How much energy can the star
    produce in total? How much energy does the star radiate away per
    second? Calibrate your expression knowing that the lifetime for
    the Sun is $1.0\cdot 10^{10}$~yr.  (N.B. In Exercise~2.5 you will derive an
    expression for the lifetime from first principles.)

  \end{enumerate}
\item  {\bf The ages of star clusters} 
  
  Hint: What is the luminosity of the most massive star which is still
  on the main-sequence? Use the mass luminosity relation to obtain the
  mass, then use your expression for the age of a star a function of
  its mass.
  



%\item {\bf Density profile}
%  \begin{enumerate}
%    \item $4\pi\rho_cr^3(\frac{1}{3} - \frac{1}{5}(r/R_*)^2$
%    \item $\frac{8}{15} \pi \rho_c R_*^3$ 
% \end{enumerate}

  
%\item {\bf Hydrostatic equilibrium (L+)}

% \begin{enumerate}
    
% \item gravitation, pressure gradient.
% \item $dm = rho(r)*dr*dA$ \\
%   $r''(t)*dm  = -G m(r) dm/ r^2 + P(r)dA - P(r + dr)dA $
   
% \item Hint: take limit for $dr$ \arrow$ 0$.
   
% \item Hint $P(R) = 0$ and use that $r \leq R_*$.
   
% \item Hint: calculate $m(r)$, substitute it directly in the equation
%   for Hydrostatic Equilibrium and integrate from the center to the
%   surface to obtain $P_c$.
   
% \end{enumerate}

\end{enumerate}

\end{document}
