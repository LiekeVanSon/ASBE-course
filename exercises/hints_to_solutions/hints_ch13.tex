\documentclass[11pt,a4paper,fleqn]{report}

\usepackage{epsfig}
\usepackage{amsmath}
\usepackage{amssymb}
\usepackage{txfonts}

\textwidth 15.5cm
\textheight 23.0cm
\oddsidemargin 0cm
\evensidemargin 0cm
\topmargin 0cm

\pagestyle{empty}

\newcommand{\hints}[1]{\centerline{{\large\bf 
      Stellar Evolution -- Hints to exercises -- Chapter #1}} \bigskip
\renewcommand{\labelenumi}{#1.\arabic{enumi}}}

% for making counters of style 1.1, 1.2, etc
%\renewcommand{\labelenumi}{1.\arabic{enumi}}

% symbol definitions:
\newcommand{\Msun}{\ensuremath{{M}_\odot}}
\newcommand{\Lsun}{\ensuremath{{L}_\odot}}
\newcommand{\Rsun}{\ensuremath{{R}_\odot}}
\newcommand{\Mcore}{\ensuremath{M_\mathrm{c}}}
\newcommand{\MCh}{\ensuremath{M_\mathrm{Ch}}}
\newcommand{\Teff}{\ensuremath{T_\mathrm{eff}}}
\newcommand{\lum}{\ensuremath{l}}
\newcommand{\ledd}{\ensuremath{l_\mathrm{Edd}}}
\newcommand{\Ledd}{\ensuremath{L_\mathrm{Edd}}}
\newcommand{\Prad}{\ensuremath{P_\mathrm{rad}}}
\newcommand{\Pgas}{\ensuremath{P_\mathrm{gas}}}
\newcommand{\Pion}{\ensuremath{P_\mathrm{ion}}}
\newcommand{\Prot}{\ensuremath{P_\mathrm{rot}}}
\newcommand{\GammaC}{\ensuremath{{\Gamma_\mathrm{C}}}}
\newcommand{\gammad}{\ensuremath{{\gamma_\mathrm{ad}}}}
\newcommand{\gradad}{\ensuremath{{\nabla_\mathrm{ad}}}}
\newcommand{\gradrad}{\ensuremath{{\nabla_\mathrm{rad}}}}
\newcommand{\tdyn}{\ensuremath{{\tau_\mathrm{dyn}}}}
\newcommand{\tKH}{\ensuremath{{\tau_\mathrm{KH}}}}
\newcommand{\tnuc}{\ensuremath{{\tau_\mathrm{nuc}}}}
\newcommand{\tms}{\ensuremath{{\tau_\mathrm{MS}}}}
\newcommand{\enuc}{\ensuremath{{\epsilon_\mathrm{nuc}}}}
\newcommand{\eneu}{\ensuremath{{\epsilon_\nu}}}
\newcommand{\egrav}{\ensuremath{{\epsilon_\mathrm{gr}}}}
\newcommand{\cP}{\ensuremath{{c_{\scriptstyle P}}}}
\newcommand{\cV}{\ensuremath{{c_{\scriptstyle V}}}}
\newcommand{\lmix}{\ensuremath{{\ell_\mathrm{m}}}}
\newcommand{\HP}{\ensuremath{{H_{\scriptstyle P}}}}
\newcommand{\vel}{\ensuremath{{\upsilon}}}
\newcommand{\vconv}{\ensuremath{{\vel_\mathrm{c}}}}
\newcommand{\vsound}{\ensuremath{{\vel_\mathrm{s}}}}
\newcommand{\veq}{\ensuremath{{\vel_\mathrm{eq}}}}
\newcommand{\vect}[1]{\mbox{\boldmath $#1$}}
\newcommand{\gradient}{\mbox{\boldmath $\nabla$}}

\newcommand{\Chi}{\ensuremath{\raisebox{0.4ex}{$\chi$}}}

\newcommand{\kmsMpc}{\ensuremath{%
  \mathrm{km}\,\mathrm{s}^{-1}\,\mathrm{Mpc}^{-1}}}

\newcommand{\arrow}{\ensuremath{\Rightarrow}}
\newcommand{\lsim}{\mathrel{\hbox{\rlap{\lower.55ex \hbox {$\sim$}}
 \kern-.3em \raise.4ex \hbox{$<$}}}}
\newcommand{\gsim}{\mathrel{\hbox{\rlap{\lower.55ex \hbox {$\sim$}}
 \kern-.3em \raise.4ex \hbox{$>$}}}}

\newcommand{\beq}{\begin{equation}}
\newcommand{\eeq}{\end{equation}}
\newcommand{\beqar}{\begin{eqnarray}}
\newcommand{\eeqar}{\end{eqnarray}}
\newcommand{\der}{\ensuremath{\mathrm{d}}}
\newcommand{\dd}{\partial}

\newcommand{\half}{\ensuremath{{\textstyle\frac{1}{2}}}}
\newcommand{\third}{\ensuremath{{\textstyle\frac{1}{3}}}}
\newcommand{\onesixth}{\ensuremath{{\textstyle\frac{1}{6}}}}

\newcommand{\amu}{\ensuremath{m_\mathrm{u}}}
\newcommand{\e}{\ensuremath{\mathrm{e}}}
\newcommand{\n}{\ensuremath{\mathrm{n}}}
\newcommand{\p}{\ensuremath{\mathrm{p}}}
\renewcommand{\H}[1]{\ensuremath{{^{#1}\mathrm{H}}}}
\newcommand{\D}{\ensuremath{\mathrm{D}}}
\newcommand{\He}[1]{\ensuremath{{^{#1}\mathrm{He}}}}
\newcommand{\Li}[1]{\ensuremath{{^{#1}\mathrm{Li}}}}
\newcommand{\Be}[1]{\ensuremath{{^{#1}\mathrm{Be}}}}
\newcommand{\B}[1]{\ensuremath{{^{#1}\mathrm{B}}}}
\newcommand{\C}[1]{\ensuremath{{^{#1}\mathrm{C}}}}
\newcommand{\N}[1]{\ensuremath{{^{#1}\mathrm{N}}}}
\renewcommand{\O}[1]{\ensuremath{{^{#1}\mathrm{O}}}}
\newcommand{\F}[1]{\ensuremath{{^{#1}\mathrm{F}}}}
\newcommand{\Na}[1]{\ensuremath{{^{#1}\mathrm{Na}}}}
\newcommand{\Ne}[1]{\ensuremath{{^{#1}\mathrm{Ne}}}}
\newcommand{\Mg}[1]{\ensuremath{{^{#1}\mathrm{Mg}}}}
\newcommand{\Al}[1]{\ensuremath{{^{#1}\mathrm{\!Al}}}}
\newcommand{\Si}[1]{\ensuremath{{^{#1}\mathrm{Si}}}}
\newcommand{\Fe}[1]{\ensuremath{{^{#1}\mathrm{Fe}}}}
\newcommand{\el}[2]{\ensuremath{{^{#2}\mathrm{#1}}}}
\newcommand{\avg}[1]{\ensuremath{\langle#1\rangle}}
\newcommand{\rate}[1]{\ensuremath{r_\mathrm{#1}}}
\newcommand{\sv}[1]{\ensuremath{\langle\sigma v\rangle_{#1}}}
\newcommand{\svpp}{\ensuremath{\langle\sigma v\rangle_\mathrm{pp}}}
\newcommand{\svpD}{\ensuremath{\langle\sigma v\rangle_\mathrm{pD}}}
\newcommand{\life}[2]{\ensuremath{\tau_\mathrm{#1}(\mathrm{#2})}}
\newcommand{\fpp}[1]{\ensuremath{f_\mathrm{pp#1}}}


\begin{document}

\hints{13}

\begin{enumerate}


\item {\bf Energy budget of core-collapse supernovae}

  \begin{enumerate}

  \item $E_{\rm gr} \approx \displaystyle \frac{G M_c^2}{R_f} = 5.2
    \times 10^{53}$ erg \quad (since $R_i \approx R_\mathrm{WD} \gg R_f$)

  \item $E_{\rm kin} = \frac{1}{2}\, M_{\rm env}\, v^2$, \quad with
    $M_{\rm env} = M_i - M_{\rm c} = 8.6\,\Msun$) ~$\Rightarrow~ v =
    3.4\times 10^8$\,cm/s (3400\,km/s)

  \item $E_{\rm ph} \approx L\, \Delta t = 4\times 10^{48}$\,erg

  \item Neutrinos take away the largest fraction of the energy
    released in the collapse.

    Number of neutrinos = $E_{\rm grav}/E_{\nu}$ (with E$_{\nu}$ the
    average energy of one neutrino) $\approx 6.5 \times 10^{58}$.

\end{enumerate} 


\item {\bf Neutrino luminosity by Si burning}

  \begin{enumerate}


  \item Number of \Fe{56} nuclei: $N$(Fe) = $\displaystyle
    \frac{M_{c}}{m_{\rm Fe}} = 4.3\times 10^{55}$.
    \[
    L_{\nu,\mathrm{Si}} = \frac{N(\mathrm{Fe}) \times 
      E_{\rm nuc,\mathrm{Fe}}}{t} = 4 \times 10^{45}\, \rm erg/s.
    \]
    The neutrinos in a supernova are emitted during about 10 seconds
    $\Rightarrow$
    \[
    L_{\nu,\mathrm{SN1987A}} = \frac{E_{\rm grav}}{10\,\mathrm{sec}} 
    = 5.2 \times 10^{52}\, \rm erg/s.
    \]

  \item $L_\nu = 4\pi d^2\, F_\nu$ (with $F_\nu$ the neutrino flux and
    $d$ the distance between the object and the observer).
    If the neutrino flux measured on earth is equal for both cases,
    \[
    \frac{L_{\rm SN1987A}}{L_{\rm Si}} = 
    \frac{d^2_{\rm SN1987A}}{d^2_{\rm Si}} ~\Rightarrow~ 
    d_{\rm Si} = 14\,\mathrm{pc}.
    \] 

\end{enumerate} 


\item {\bf Carbon ignition of a white dwarf}

  To compute the radius and gravitational potential energy of the
  white dwarf, use the fact that a WD close to the Chandrasekhar limit
  is approximately described by a $n=3$ polytrope (relativistic
  degeneracy!). 

  Chapter 4, Table 4.1: \quad $\rho_c = 54.2\, \bar{\rho} = 54.2\,
  \displaystyle \frac{M_{\rm WD}}{\frac{4\pi}{3}R_{\rm WD}^3}$, \\[1ex]
  with $M_{\rm WD} = M_\mathrm{Ch} \approx 1.4\,\Msun$ and $\rho_c =
  2\times10^9$\,g/cm$^3~~\Rightarrow~ R_{\rm WD} = 2.6\times10^8$\,cm
  \[
  E_{\rm gr} = -\frac{3}{5-n}\,\frac{G M_{\rm WD}^2}{R_\mathrm{WD}}
  \approx -\frac{3}{2}\,\frac{G M_{\rm Ch}^2}{R_\mathrm{WD}}
  = -3.0\times 10^{51}\,{\rm erg} 
  \]
  
  To compute $E_{\rm nuc}$, consider that:
  \begin{itemize}
  \item
    Fusion of \C{12} into \el{Ni}{56} releases ~ $Q_1 = [m(\el{Ni}{56}) -
    \frac{56}{12}m(\C{12})]\,c^2$ ~ per \el{Ni}{56} nucleus.
  \item
    Fusion of \O{16} into \el{Ni}{56} releases ~ $Q_2 = [m(\el{Ni}{56}) -
    \frac{56}{16}m(\O{16})]\,c^2$ ~ per \el{Ni}{56} nucleus.
  \item
    Total number of \el{Ni}{56} nuclei created is 
    $M_\mathrm{Ch}/m(\el{Ni}{56})$.
  \end{itemize}
  With equal mass fractions of C and O:
  \[
  E_{\rm nuc} = \frac{M_\mathrm{Ch}}{m(\el{Ni}{56})}\times 0.5 (Q_1 + Q_2) 
  = 2.2\times 10^{51}\,\mathrm{erg}.
  \]

  Hence we conclude that $E_{\rm nuc} < |E_{\rm gr}|$. 

  However, keep in mind that the \emph{total} binding energy of the WD
  is $E_{\rm tot} = E_{\rm gr} + E_{\rm int}$. In this case $E_{\rm
    int}$ is provided by the Fermi energies of the degenerate
  electrons in the white dwarf. If this is taken into account, then
  $|E_{\rm tot}| \ll |E_{\rm gr}|$ and $E_{\rm nuc} > |E_{\rm tot}|$,
  i.e.\ enough energy is generated to unbind the WD.

  
\end{enumerate} 


\end{document}
