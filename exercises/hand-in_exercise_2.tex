\documentclass[a4paper,11pt]{article}
\usepackage{txfonts}
\usepackage{fancybox}
\usepackage{epsfig}
\textwidth 16cm
\textheight 24cm
\oddsidemargin 0.0cm
\topmargin -1cm

\newcommand{\Msun}{\ensuremath{{M}_\odot}}
\newcommand{\Rsun}{\ensuremath{{R}_\odot}}
\newcommand{\Lsun}{\ensuremath{{L}_\odot}}
\newcommand{\gammad}{\ensuremath{\gamma_\mathrm{a}}}
\newcommand\der{\ensuremath{\mathrm{d}}}

%\newcounter{exercise}
%\renewcommand{\theexercise}{\arabic{exercise}}
%\newcommand{\exercise}[1]{\refstepcounter{exercise}
%    \section*{Exercise \theexercise:~~ {\it #1}}}


\begin{document}

\begin{center}
{\Large\bf Advanced Stellar and Binary Evolution} \\ [1ex]
{\large hand-in exercise, week 4}

\end{center}

\subsection*{1.~~Evolution of AGB stars}
\label{ex:AGB-evol}

The luminosity of an AGB star is related to its core mass via the
Paczynski relation~(eq.~11.1),
\[
    {L\over\Lsun}=5.9\times 10^4 \left({M_c\over\Msun} - 0.52\right).
\]
Assume that a star enters the AGB with a luminosity of $3.0\times10^3~\Lsun$, effective temperature of 3000~K and a total mass of 2~$\Msun$.
    
\begin{enumerate}

\item[(a)] Mass is gradually added to the core by nuclear burning in the H- and He-burning shells. Assuming for the moment that both shells burn at a steady rate (i.e.\ ignoring thermal pulses), and assuming that the envelope of the star has a hydrogen mass fraction $X=0.7$, show that the core mass grows at a rate of about 
\[ \dot M_c = 1.2\times 10^{-11} \left(L\over\Lsun\right)~\Msun/\mathrm{yr}. \]

(Hint: see Sect.~6.4.1-2 for the energy released by H- and He-burning. You can assume He-burning occurs by the $3\alpha$ reaction.)


\item[(b)] Derive an expression for the luminosity as a function of time
  after the star entered the AGB phase.
    
\item[(c)] Assume that $T_{\rm eff}$ remains constant at 3000~K and derive an
  expression for the radius as a function of time.
    
\item[(d)] Derive an expression for the core mass as a function of time.
    
\end{enumerate}
    
\noindent
The masses of white dwarfs and the luminosity on the tip of the AGB
are mostly determined by \emph{mass loss} during the AGB phase. The mass-loss 
rate is uncertain, but for this exercise assume it is given by the Reimers
relation, eq.~(10.3),
\[
      {\dot M\over\Msun\,\mathrm{yr}^{-1}}= 4\times 10^{-13}\eta
      \left({L\over\Lsun}\right)
      \left({R\over\Rsun}\right)\left({\Msun\over M}\right),      
\]
with $\eta\approx3$ for AGB stars. 
      
\begin{enumerate}

\item[(e)] Derive an expression for the mass of the AGB star as a function of
  time, using $L(t)$ and $R(t)$ from the previous questions.
  (Hint: $-\dot{M} M = \frac{1}{2}\der(M^2)/\der t$).
      
\item[(f)] Use the expression from (e) and the one for $M_c(t)$ from
  (d) to derive:
  \begin{itemize}

  \item the time when the star leaves the AGB ($M_\mathrm{env}\simeq
    0$),

  \item the luminosity at the tip of the AGB,

  \item the mass of the resulting white dwarf.

  \end{itemize}
  This requires a numerical solution of a simple equation, but you can
  also solve it graphically by plotting the two relations.
     
\item[(g)] In the above calculations, we have ignored the fact that \emph{thermal pulses} occur during the AGB, i.e.\ we have assumed that the AGB star undergoes steady H- and He-shell burning.  Discuss qualitatively how the answers to the above questions will change if thermal pulses are taken into account, in particular the \emph{dredge-up} episodes that follow each thermal pulse.
	
\end{enumerate} 
            
\end{document} 

