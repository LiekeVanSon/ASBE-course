\documentclass[a4paper,11pt]{article}
\usepackage{txfonts}
\usepackage{fancybox}
\usepackage{epsfig}
\textwidth 16cm
\textheight 24cm
\oddsidemargin 0.0cm
\topmargin -1cm

\newcommand{\Msun}{\ensuremath{{M}_\odot}}
\newcommand{\Rsun}{\ensuremath{{R}_\odot}}
\newcommand{\Lsun}{\ensuremath{{L}_\odot}}
\newcommand{\gammad}{\ensuremath{\gamma_\mathrm{a}}}
\newcommand\der{\ensuremath{\mathrm{d}}}

%\newcounter{exercise}
%\renewcommand{\theexercise}{\arabic{exercise}}
%\newcommand{\exercise}[1]{\refstepcounter{exercise}
%    \section*{Exercise \theexercise:~~ {\it #1}}}


\begin{document}

\begin{center}
{\Large\bf Advanced Stellar and Binary Evolution} \\ [1ex]
{\large hand-in exercise, week 2}

\end{center}

\subsection*{1.~~Nuclear power and energy transport in a main-sequence star}
\label{ex:rad-conv-cores}

Consider a hydrogen-burning star with a mass of $M=3\Msun$, luminosity $L=80\Lsun$, and an initial composition $X=0.7$ and $Z=0.02$. The nuclear energy in this star is generated only in the central 10\% of the mass, and the energy generation rate per unit mass, $\epsilon_{nuc}$, in this part of the star depends on the mass coordinate $m$ as
%\begin{equation}
\[
\epsilon_{nuc}=\epsilon_c\left( 1-\frac{m}{0.1M}\right)
\]
%\end{equation}

\begin{itemize}
\item[a)] Calculate and draw the internal luminosity profile $l(m)$ as a function of $m$. Express $\epsilon_c$ in terms of the total mass and luminosity of the star.

\item[b)] Assume for now that all the generated energy is transported by radiation. Derive an expression for the H-abundance profile as a function of mass and time, $X=X(m,t)$. What is the central value for $X$ after 100 Myr? Draw $X$ as a function of $m$ at this time.

[Hint: the energy generated by H-burning per unit mass is $Q=6.3\times10^{18} {\rm~erg~g^{-1}}$.]

\item[c)] Explain why the assumption of radiative energy transport is unlikely to be correct \emph{everywhere} in this star.

\item[d)] In reality, a part of this star is unstable to convection. Let's make the following additional assumptions: 
\begin{itemize}
\item the opacity inside the star is completely due to electron scattering, 
\item the gas inside the star behaves like a fully ionized ideal gas,
\item the contribution of radiation pressure to the total pressure is a constant 0.2\% throughout the star. 
\end{itemize}
Now use the Schwarzschild criterion, in combination with your answer to question (a), to compute the extent (in mass coordinate $m$) of the convective region.

[Hints: Use the equation for radiation pressure to simplify the expression for the radiative temperature gradient $\nabla_\mathrm{rad}$. Also take into consideration that the luminosity profile from question (a) is a piecewise function with different regimes. And finally, be very careful with using or converting units in this exercise.]

\item[e)] Inside the convective region the gas is fully mixed. Assuming the size of the convective region stays fixed in time, answer the same question as in (b) and draw the new $X$ profile as a function of $m$ at an age of 100 Myr. 

\item[f)] By which factor is the central H-burning lifetime extended as a result of convection?

\end{itemize}
	
            
\end{document} 

