\documentclass[11pt,a4paper]{article}
\usepackage[utf8]{inputenc}

\pagenumbering{arabic}
\usepackage[margin=1.0in]{geometry}
\usepackage{natbib} 
\usepackage{hyperref}
\usepackage{enumitem}

\newcommand{\todo}[1]{\textbf{\textcolor{red}{#1}}}

% define a bunch of colors for pretty bash code blocks
\usepackage{xcolor}  % for coloring
\definecolor{bgdark}{rgb}{0.27,0.27,0.27}
\definecolor{lighttext}{rgb}{0.95,0.95,0.95}
\definecolor{bashgreen}{rgb}{0.5,1.0,0.5}
\definecolor{bashblue}{rgb}{0.5,0.8,1.0}
\definecolor{bashcomment}{rgb}{0.6,0.6,0.6}
\definecolor{bashstring}{rgb}{1.0,0.6,0.6}

% For pretty code blocks
\usepackage{listings}
% Bash style 
\lstset{
  language=bash,
  basicstyle=\ttfamily\small\color{lighttext},
  backgroundcolor=\color{bgdark},
  keywordstyle=\color{bashblue},
  commentstyle=\color{bashcomment},
  stringstyle=\color{bashstring},
  numbers=left,
  numberstyle=\tiny\color{bashcomment},
  stepnumber=1,
  frame=single,
  breaklines=true,
  showstringspaces=false
}
% Python style
\definecolor{pylightbg}{RGB}{250,250,250}
\definecolor{pyblue}{RGB}{0,0,180}      % keywords
\definecolor{pygreen}{RGB}{0,150,0}     % comments
\definecolor{pyred}{RGB}{180,0,0}       % strings
\definecolor{gray}{gray}{0.5}           % line numbers
\lstdefinestyle{pythonstyle}{
  language=Python,
  basicstyle=\ttfamily\small\color{black},
  backgroundcolor=\color{pylightbg},
  keywordstyle=\color{pyblue}\bfseries,
  commentstyle=\color{pygreen}\itshape,
  stringstyle=\color{pyred},
  numbers=left,
  numberstyle=\tiny\color{gray},
  stepnumber=1,
  frame=single,
  breaklines=true,
  showstringspaces=false
}


% For making the pro-tip boxes
\usepackage[most]{tcolorbox}
% define colors for the pro-tip boxes
\definecolor{protipbg}{HTML}{ebf0f2}    % 
\definecolor{protipborder}{HTML}{B0CAD4} % 
\definecolor{protiptext}{HTML}{003049}  % dark slate
\tcbset{
  protipbox/.style={
    colback=protipbg,
    colframe=protipborder,
    coltext=protiptext,
    fonttitle=\bfseries,
    title=Pro Tip,
    rounded corners,
    boxrule=0.pt,
    left=6pt,
    right=6pt,
    top=5pt,
    bottom=5pt
  }
}


%%%%%%%%%%%%%%%%%%%%%%%%%%%%%%%%%%%%%%%%%%%%
\begin{document}


\title{
    \textbf{\texttt{MESA} tutorial: Session 2} \\
    \textbf{\Large Intermediate mass stars and rotation}
}
\date{}
\maketitle
\vspace{-2cm}


\section{Overshooting in an intermediate-mass star}

In this first exercise, we will investigate the effect of overshooting. In order to do this, we evolve an intermediate-mass star for different levels of convective overshooting.


\begin{enumerate}

\item \textbf{Download and set up the work folder.} 
Instead of starting from the beginning and copying the model work directory like last time to edit, a \texttt{MESA} model has largely been constructed for you already. 


\begin{enumerate} 

    \item Download the tar file from Brightspace and unpack it. This can be done using the archive manager by extracting the contents to your desired location on the \texttt{scratch} disc, or using the terminal by typing \verb|tar -xvf session2.tar| in the directory where you downloaded the tar file and then moving the directory to your desired location on the \texttt{scratch} disc using a \texttt{cp -R} command. This file contains a \texttt{MESA} work folder in which we will work. 
    
    \item Once you open the \verb|inlist_project| file, you will find a number of controls available for the model. First, choose a mass for the star between 2.5 M$_\odot$ and 10 M$_\odot$, and modify the inlist accordingly. Note that the stopping condition allows the star to evolve up to the end of core helium burning. You may further notice several lines commented out concerning Convective overshooting. We will run several models for different values of the overshooting parameter. 

    \item First, compile (\verb|./mk|) and run (\verb|./rn|) the model without any overshooting. 
    Use the various \texttt{PGSTAR} windows to follow and understand the evolution of this star. One of the \texttt{PGSTAR} windows is a Kippenhahn diagram, which shows information about the structure of the star as it evolves. Also try to understand this plot.
    %\emph{Note:} You can zoom in and out on the time axis of the Kippenhahn plot by changing \verb|Kipp_max_width| in \verb|inlist_pgstar| and saving the file. You may also have to change the limits of the HRD plot.


    \item After the model finishes, create a new copy of the work folder and rename the \textbf{old} folder to an appropriate name (e.g. identify it by the mass and overshoot value used, like \verb|M3_0ov0| for mass 3 and overshoot 0). 
    In the new work folder, uncomment the lines relating to convective overshooting in your inlist and change the overshooting parameter \verb|overshoot_f(1)| to 0.25. 
    Look for the meaning of this and other overshooting parameters in the file \verb|$MESA_DIR/star/defaults/controls.defaults| or by checking \href{https://docs.mesastar.org/en/latest/reference/controls.html}{the documentation online}.
    \footnote{Note that the documentation online is for the latest version of \texttt{MESA}, which may differ slightly from the version you are using. The bottom right of the documentation page shows which version of the docs you are viewing, but note this only dates back to version r15140, which is when MESA was migrated to GitHub.
    \texttt{\$MESA\_DIR/star/defaults/controls.defaults} will always show the information for the correct version for your installation.}
    Then compile and run the code again.  Repeat this process for an overshooting parameter of 0.5.

\end{enumerate}


% \begin{tcolorbox}[protipbox]
% Can you get the same behaviour as above but using the \texttt{log\_directory = `LOGS'} control option?
% \end{tcolorbox}


\item \textbf{Inspecting the results. }
We can now analyse the data in the 3 history files. To do this, go to the website under the tab `using \texttt{MESA} output'. Go to the section on \href{https://docs.mesastar.org/en/latest/using_mesa/output.html#plotting-mesa-output}{MESA Reader}. 
%
You can either pip install Mesa Reader by running: \todo{discuss virtual envs?}
  \begin{lstlisting}
  pip install mesa_reader
  \end{lstlisting}
Or directly download the code from the project's \href{https://github.com/wmwolf/py_mesa_reader}{Github repository} by clicking on the green \texttt{Code} button and choosing \texttt{Download ZIP}. 
%
\texttt{MESA Reader} is a \texttt{Python} framework with which one can easily plot data from the history and profile files. Read through the section on \texttt{MESA Reader} on the website to get familiar with its capabilities. 
Extract the folder titled \texttt{mesa\_reader} from the zip file and place it somewhere in your \texttt{home} directory where you will keep your course-related plots (e.g. \verb|~/MESA/codes|)\footnote{Note that the symbol \texttt{\~} points to your home directory.}. 
In this plotting folder, create your own \texttt{Python} script and import \texttt{MESA Reader} as follows:
\begin{lstlisting}[style=pythonstyle]
 import mesa_reader as mr
\end{lstlisting}

Note some other useful imports for calculating and plotting purposes:
\begin{lstlisting}[style=pythonstyle]
  import numpy as np
  import matplotlib.pyplot as plt
\end{lstlisting}

\todo{come back to this when I have proper jupyterhub access}
You can use \href{{https://jupyterhub.science.ru.nl/}}{Jupyter Notebook}, which you can access via and logging in with your science account. 
Make sure your notebook is inside your plotting folder as to be able to access \texttt{MESA Reader}.
%
% , or \texttt{Spyder3}\footnote{When running scripts in \texttt{Spyder}, you need to change the working directory as to have access to \texttt{MESA Reader} by going to \texttt{Run}, \texttt{Configuration per file...}, and changing \texttt{Working Directory settings} by putting in the correct path to your plotting folder with the \texttt{The following directory} option.}. 
%
To read in the data, you need to move the \texttt{history.data} files from your \texttt{scratch} directory to somewhere in your plotting folder, for example inside a folder called \texttt{data} and there inside a folder called after its run (e.g. \texttt{M3.0ov0}). For this example, you use the following line in your \texttt{Python} script:

\begin{lstlisting}[style=pythonstyle]
f0_hist_data = mr.MesaData('data/M3.0ov0/LOGS/history.data')
\end{lstlisting}


\begin{tcolorbox}[protipbox]
As explained in section 2.2 of the first tutorial, it is a very good idea to copy your work folder from the \verb|/scratch| directory to your home directory, after the \texttt{MESA} run has finished. This allows you to analyse your results using \texttt{MESA Reader} from any computer in the Faculty, not just the computer you ran your \texttt{MESA} models on! 
\end{tcolorbox}


\emph{\textbf{Note:}} 
It is also a good idea to organise your \texttt{MESA} work folders in a logical directory structure, in order to not make your home directory a mess. Just make sure you change the path in your \texttt{Python} script, such that it points to the right file. \\

To invoke the columns you want from the read-in data, in this case \texttt{f0\_hist\_data}, you use \texttt{f0\_hist\_data.X}, where X is the name of one of the columns of data inside \texttt{history.data}. The names of the columns can be found inside \texttt{history.data}, or you can print the available column names by typing: 

\begin{lstlisting}[style=pythonstyle]
print(f0_hist_data.bulk_names)
\end{lstlisting}

\begin{enumerate}

  \item Make an HR diagram containing the 3 models. What changes do you see in the main-sequence evolution? What changes appear in the evolution \emph{after} the main sequence? Can you explain these changes?

  \item Make a plot of $\rho_c$ vs $T_c$. How do the evolution tracks in this diagram change for different levels of overshooting?

  \item Construct a plot of the central helium abundance vs age for all 3 models and explain your findings. 

  \item By what fraction is the main sequence lifetime increased for an overshooting parameter of 0.25 compared to the model without overshooting? By what fraction does the \emph{helium burning} lifetime change? Compare your findings with your neighbours who, hopefully, have chosen a star of different mass.

\end{enumerate}

\end{enumerate}

%%%%%%%%%%%%%%%%%%%%%%%%%%%%%%%
\section{\todo{Rotation}}
\todo{add a (short) exercise on rotation based on my hackathon from last year}

Note how in some way rotation can mimic the effects of overshooting, as it also leads to mixing beyond the convective core.

\begin{itemize}
    \item Investigate the effects of rotation on the evolution of intermediate-mass stars.
    \item Compare the evolutionary tracks of non-rotating and rotating models in the HR diagram.
    \item Analyze how rotation affects the main-sequence lifetime and the subsequent evolution of the star.
\end{itemize}



% just to cite something \cite{Paxton2011}
% \bibliographystyle{plainnat}
% \bibliography{gmlib.bib}

\end{document}
